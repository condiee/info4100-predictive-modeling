% Options for packages loaded elsewhere
\PassOptionsToPackage{unicode}{hyperref}
\PassOptionsToPackage{hyphens}{url}
%
\documentclass[
]{article}
\usepackage{lmodern}
\usepackage{amssymb,amsmath}
\usepackage{ifxetex,ifluatex}
\ifnum 0\ifxetex 1\fi\ifluatex 1\fi=0 % if pdftex
  \usepackage[T1]{fontenc}
  \usepackage[utf8]{inputenc}
  \usepackage{textcomp} % provide euro and other symbols
\else % if luatex or xetex
  \usepackage{unicode-math}
  \defaultfontfeatures{Scale=MatchLowercase}
  \defaultfontfeatures[\rmfamily]{Ligatures=TeX,Scale=1}
\fi
% Use upquote if available, for straight quotes in verbatim environments
\IfFileExists{upquote.sty}{\usepackage{upquote}}{}
\IfFileExists{microtype.sty}{% use microtype if available
  \usepackage[]{microtype}
  \UseMicrotypeSet[protrusion]{basicmath} % disable protrusion for tt fonts
}{}
\makeatletter
\@ifundefined{KOMAClassName}{% if non-KOMA class
  \IfFileExists{parskip.sty}{%
    \usepackage{parskip}
  }{% else
    \setlength{\parindent}{0pt}
    \setlength{\parskip}{6pt plus 2pt minus 1pt}}
}{% if KOMA class
  \KOMAoptions{parskip=half}}
\makeatother
\usepackage{xcolor}
\IfFileExists{xurl.sty}{\usepackage{xurl}}{} % add URL line breaks if available
\IfFileExists{bookmark.sty}{\usepackage{bookmark}}{\usepackage{hyperref}}
\hypersetup{
  pdftitle={Group Project: Early Alert with LMS Data},
  pdfauthor={{[}{[}Jacob Bee Ho Brown (jdb393), Emma Condie (erc97), Ellie Garell (emg223), Sanithia Edwards (se273), Yue Ji (yj359), Julia Lehman (jsl357), Heitung Sun (hs835){]}{]}},
  hidelinks,
  pdfcreator={LaTeX via pandoc}}
\urlstyle{same} % disable monospaced font for URLs
\usepackage[margin=1in]{geometry}
\usepackage{color}
\usepackage{fancyvrb}
\newcommand{\VerbBar}{|}
\newcommand{\VERB}{\Verb[commandchars=\\\{\}]}
\DefineVerbatimEnvironment{Highlighting}{Verbatim}{commandchars=\\\{\}}
% Add ',fontsize=\small' for more characters per line
\usepackage{framed}
\definecolor{shadecolor}{RGB}{248,248,248}
\newenvironment{Shaded}{\begin{snugshade}}{\end{snugshade}}
\newcommand{\AlertTok}[1]{\textcolor[rgb]{0.94,0.16,0.16}{#1}}
\newcommand{\AnnotationTok}[1]{\textcolor[rgb]{0.56,0.35,0.01}{\textbf{\textit{#1}}}}
\newcommand{\AttributeTok}[1]{\textcolor[rgb]{0.77,0.63,0.00}{#1}}
\newcommand{\BaseNTok}[1]{\textcolor[rgb]{0.00,0.00,0.81}{#1}}
\newcommand{\BuiltInTok}[1]{#1}
\newcommand{\CharTok}[1]{\textcolor[rgb]{0.31,0.60,0.02}{#1}}
\newcommand{\CommentTok}[1]{\textcolor[rgb]{0.56,0.35,0.01}{\textit{#1}}}
\newcommand{\CommentVarTok}[1]{\textcolor[rgb]{0.56,0.35,0.01}{\textbf{\textit{#1}}}}
\newcommand{\ConstantTok}[1]{\textcolor[rgb]{0.00,0.00,0.00}{#1}}
\newcommand{\ControlFlowTok}[1]{\textcolor[rgb]{0.13,0.29,0.53}{\textbf{#1}}}
\newcommand{\DataTypeTok}[1]{\textcolor[rgb]{0.13,0.29,0.53}{#1}}
\newcommand{\DecValTok}[1]{\textcolor[rgb]{0.00,0.00,0.81}{#1}}
\newcommand{\DocumentationTok}[1]{\textcolor[rgb]{0.56,0.35,0.01}{\textbf{\textit{#1}}}}
\newcommand{\ErrorTok}[1]{\textcolor[rgb]{0.64,0.00,0.00}{\textbf{#1}}}
\newcommand{\ExtensionTok}[1]{#1}
\newcommand{\FloatTok}[1]{\textcolor[rgb]{0.00,0.00,0.81}{#1}}
\newcommand{\FunctionTok}[1]{\textcolor[rgb]{0.00,0.00,0.00}{#1}}
\newcommand{\ImportTok}[1]{#1}
\newcommand{\InformationTok}[1]{\textcolor[rgb]{0.56,0.35,0.01}{\textbf{\textit{#1}}}}
\newcommand{\KeywordTok}[1]{\textcolor[rgb]{0.13,0.29,0.53}{\textbf{#1}}}
\newcommand{\NormalTok}[1]{#1}
\newcommand{\OperatorTok}[1]{\textcolor[rgb]{0.81,0.36,0.00}{\textbf{#1}}}
\newcommand{\OtherTok}[1]{\textcolor[rgb]{0.56,0.35,0.01}{#1}}
\newcommand{\PreprocessorTok}[1]{\textcolor[rgb]{0.56,0.35,0.01}{\textit{#1}}}
\newcommand{\RegionMarkerTok}[1]{#1}
\newcommand{\SpecialCharTok}[1]{\textcolor[rgb]{0.00,0.00,0.00}{#1}}
\newcommand{\SpecialStringTok}[1]{\textcolor[rgb]{0.31,0.60,0.02}{#1}}
\newcommand{\StringTok}[1]{\textcolor[rgb]{0.31,0.60,0.02}{#1}}
\newcommand{\VariableTok}[1]{\textcolor[rgb]{0.00,0.00,0.00}{#1}}
\newcommand{\VerbatimStringTok}[1]{\textcolor[rgb]{0.31,0.60,0.02}{#1}}
\newcommand{\WarningTok}[1]{\textcolor[rgb]{0.56,0.35,0.01}{\textbf{\textit{#1}}}}
\usepackage{graphicx,grffile}
\makeatletter
\def\maxwidth{\ifdim\Gin@nat@width>\linewidth\linewidth\else\Gin@nat@width\fi}
\def\maxheight{\ifdim\Gin@nat@height>\textheight\textheight\else\Gin@nat@height\fi}
\makeatother
% Scale images if necessary, so that they will not overflow the page
% margins by default, and it is still possible to overwrite the defaults
% using explicit options in \includegraphics[width, height, ...]{}
\setkeys{Gin}{width=\maxwidth,height=\maxheight,keepaspectratio}
% Set default figure placement to htbp
\makeatletter
\def\fps@figure{htbp}
\makeatother
\setlength{\emergencystretch}{3em} % prevent overfull lines
\providecommand{\tightlist}{%
  \setlength{\itemsep}{0pt}\setlength{\parskip}{0pt}}
\setcounter{secnumdepth}{-\maxdimen} % remove section numbering

\title{Group Project: Early Alert with LMS Data}
\usepackage{etoolbox}
\makeatletter
\providecommand{\subtitle}[1]{% add subtitle to \maketitle
  \apptocmd{\@title}{\par {\large #1 \par}}{}{}
}
\makeatother
\subtitle{INFO 4100 Learning Analytics}
\author{{[}{[}Jacob Bee Ho Brown (jdb393), Emma Condie (erc97), Ellie Garell
(emg223), Sanithia Edwards (se273), Yue Ji (yj359), Julia Lehman
(jsl357), Heitung Sun (hs835){]}{]}}
\date{}

\begin{document}
\maketitle

\hypertarget{introduction}{%
\section{Introduction}\label{introduction}}

\textbf{Goals:} The goal of this project is to learn how to work with
raw Learning Management System (LMS) data and apply some of the
prediction skills you have learned so far. You will develop a one-day
early warning system for students who miss a graded submission. I am
sharing with you an export of the class's edX log data thus far. I have
anonymized the dataset and performed minimal data cleaning, leaving
plenty of real-world messiness for you to tackle here. As always, you
should start by getting to know the datasets. In this case, you should
be able to really understand what is going on because it is YOUR data.
In fact, you can navigate to the relevant pages on edX to see what
page/action the data refers to.

\textbf{Group Project:} This is a group project and I expect you to work
as a team to come up with the best possible prediction accuracy. Your
team will submit one common solution (note that EACH team member will
need to submit the knitted Word doc on edx to get credit like with the
first group project).

\textbf{Try Your Best:} All members of the TWO teams that achieve the
highest F1 scores will receive an extra credit point, and their
solutions will be featured. To be eligible, your prediction problem
needs to be set up correctly (i.e.~everything else needs to be correct).

\hypertarget{step-1-understand-the-data}{%
\section{Step 1: Understand the data}\label{step-1-understand-the-data}}

There are three datasets which can be connected using the hash\_id
column (a hashed version of the user id) and I am giving you links to
the official documentation which you should read to understand the data
better:

\begin{enumerate}
\def\labelenumi{\arabic{enumi}.}
\tightlist
\item
  Clickstream data (1 row per student per action):
  \href{https://edx.readthedocs.io/projects/devdata/en/stable/internal_data_formats/tracking_logs.html\#tracking-logs}{click
  for documentation}
\item
  Module States (1 row per student per accessed content): original name
  \href{https://edx.readthedocs.io/projects/devdata/en/stable/internal_data_formats/sql_schema.html\#courseware-studentmodule}{courseware-studentmodule
  (click for doumentation)}
\item
  Assessment grades (1 row per assessment per student)
\end{enumerate}

I have already converted date-time objects into a numeric
\texttt{timestamp} for you.

To look up what pages URLs refer to (works for browser events, not
server events), you can paste the URL into your browser. This should
work for most URLs. I recommend doing this to be able to engineer more
meaningful features.

\emph{Question 1:} In the space below, explore each dataset using
\texttt{head()}, \texttt{n\_distinct(data\$some\_id)},
\texttt{summary()}, \texttt{table(data\$column)}. You can also plot the
distribution of variables with histograms or boxplots. Check out the
data documentation linked above to understand the meaning of each
column.

\begin{Shaded}
\begin{Highlighting}[]
\CommentTok{############################################### }
\CommentTok{###### }\RegionMarkerTok{BEGIN}\CommentTok{ INPUT: Explore each dataset ###### }
\CommentTok{###############################################}

\CommentTok{# Exploring Clickstreams}
\KeywordTok{head}\NormalTok{(cl)}
\KeywordTok{n_distinct}\NormalTok{(cl}\OperatorTok{$}\NormalTok{hash_id)}
\KeywordTok{n_distinct}\NormalTok{(cl}\OperatorTok{$}\NormalTok{survey_id)}
\KeywordTok{n_distinct}\NormalTok{(cl}\OperatorTok{$}\NormalTok{time)}
\KeywordTok{n_distinct}\NormalTok{(cl}\OperatorTok{$}\NormalTok{name)}
\KeywordTok{n_distinct}\NormalTok{(cl}\OperatorTok{$}\NormalTok{event_type)}
\KeywordTok{n_distinct}\NormalTok{(cl}\OperatorTok{$}\NormalTok{referer)}
\KeywordTok{n_distinct}\NormalTok{(cl}\OperatorTok{$}\NormalTok{page)}
\KeywordTok{n_distinct}\NormalTok{(cl}\OperatorTok{$}\NormalTok{event_source)}
\KeywordTok{n_distinct}\NormalTok{(cl}\OperatorTok{$}\NormalTok{event)}
\KeywordTok{n_distinct}\NormalTok{(cl}\OperatorTok{$}\NormalTok{timestamp)}
\CommentTok{#summary(cl)}
\KeywordTok{table}\NormalTok{(cl}\OperatorTok{$}\NormalTok{hash_id)}
\CommentTok{#table(cl$survey_id)}
\CommentTok{#table(cl$time)}
\KeywordTok{table}\NormalTok{(cl}\OperatorTok{$}\NormalTok{name)}
\KeywordTok{table}\NormalTok{(cl}\OperatorTok{$}\NormalTok{event_type)}
\KeywordTok{table}\NormalTok{(cl}\OperatorTok{$}\NormalTok{referer)}
\KeywordTok{table}\NormalTok{(cl}\OperatorTok{$}\NormalTok{page)}
\KeywordTok{table}\NormalTok{(cl}\OperatorTok{$}\NormalTok{event_source)}
\NormalTok{cl}\OperatorTok{$}\NormalTok{event}
\KeywordTok{table}\NormalTok{(cl}\OperatorTok{$}\NormalTok{timestamp)}

\CommentTok{# Exploring Module States}
\KeywordTok{head}\NormalTok{(m)}
\KeywordTok{n_distinct}\NormalTok{(m}\OperatorTok{$}\NormalTok{hash_id)}
\KeywordTok{n_distinct}\NormalTok{(m}\OperatorTok{$}\NormalTok{module_type)}
\KeywordTok{n_distinct}\NormalTok{(m}\OperatorTok{$}\NormalTok{grade)}
\KeywordTok{n_distinct}\NormalTok{(m}\OperatorTok{$}\NormalTok{created)}
\KeywordTok{n_distinct}\NormalTok{(m}\OperatorTok{$}\NormalTok{modified)}
\KeywordTok{n_distinct}\NormalTok{(m}\OperatorTok{$}\NormalTok{max_grade)}
\KeywordTok{n_distinct}\NormalTok{(m}\OperatorTok{$}\NormalTok{module_id)}
\KeywordTok{n_distinct}\NormalTok{(m}\OperatorTok{$}\NormalTok{created_timestamp)}
\KeywordTok{n_distinct}\NormalTok{(m}\OperatorTok{$}\NormalTok{modified_timestamp)}
\CommentTok{#summary(m)}
\CommentTok{#table(cl$hash_id)}
\KeywordTok{table}\NormalTok{(m}\OperatorTok{$}\NormalTok{module_type)}
\KeywordTok{table}\NormalTok{(m}\OperatorTok{$}\NormalTok{grade)}
\KeywordTok{table}\NormalTok{(m}\OperatorTok{$}\NormalTok{created)}
\KeywordTok{table}\NormalTok{(m}\OperatorTok{$}\NormalTok{modified)}
\KeywordTok{table}\NormalTok{(m}\OperatorTok{$}\NormalTok{max_grade)}
\KeywordTok{table}\NormalTok{(m}\OperatorTok{$}\NormalTok{module_id)}
\KeywordTok{table}\NormalTok{(m}\OperatorTok{$}\NormalTok{created_timestamp)}
\KeywordTok{table}\NormalTok{(m}\OperatorTok{$}\NormalTok{modified_timestamp)}

\CommentTok{# Exploring Assessment grades}
\CommentTok{# add code here}
\KeywordTok{head}\NormalTok{(a)}
\KeywordTok{n_distinct}\NormalTok{(a}\OperatorTok{$}\NormalTok{hash_id)}
\KeywordTok{n_distinct}\NormalTok{(a}\OperatorTok{$}\NormalTok{usage_key)}
\KeywordTok{n_distinct}\NormalTok{(a}\OperatorTok{$}\NormalTok{earned_graded)}
\KeywordTok{n_distinct}\NormalTok{(a}\OperatorTok{$}\NormalTok{first_attempted)}
\KeywordTok{n_distinct}\NormalTok{(a}\OperatorTok{$}\NormalTok{created)}
\KeywordTok{n_distinct}\NormalTok{(a}\OperatorTok{$}\NormalTok{created_timestamp)}
\KeywordTok{n_distinct}\NormalTok{(a}\OperatorTok{$}\NormalTok{modified_timestamp)}
\KeywordTok{n_distinct}\NormalTok{(a}\OperatorTok{$}\NormalTok{first_attempted_timestamp)}
\CommentTok{#summary(a)}
\CommentTok{#table(a$hash_id)}
\CommentTok{#table(a$usage_key)}
\KeywordTok{table}\NormalTok{(a}\OperatorTok{$}\NormalTok{earned_graded)}
\KeywordTok{table}\NormalTok{(a}\OperatorTok{$}\NormalTok{first_attempted)}
\KeywordTok{table}\NormalTok{(a}\OperatorTok{$}\NormalTok{created)}
\KeywordTok{table}\NormalTok{(a}\OperatorTok{$}\NormalTok{created_timestamp)}
\KeywordTok{table}\NormalTok{(a}\OperatorTok{$}\NormalTok{modified_timestamp)}
\KeywordTok{table}\NormalTok{(a}\OperatorTok{$}\NormalTok{first_attempted_timestamp)}

\CommentTok{###############################################}
\CommentTok{###############################################}
\end{Highlighting}
\end{Shaded}

You may notice that it would be helpful to combine the information about
grades and time of first attempt with the module state data. Below I
make this join for you. See that only `sequential' modules have grade
data associated with them. The boxplot shows when the different
sequentials (containing problems) were attempted. This gives you an idea
of the order of problems in the course.

\begin{Shaded}
\begin{Highlighting}[]
\NormalTok{ma =}\StringTok{ }\NormalTok{m }\OperatorTok\StringTok{ }\KeywordTok{left_join}\NormalTok{(}
\NormalTok{    a }\OperatorTok\StringTok{ }\KeywordTok{select}\NormalTok{(hash_id}\OperatorTok{:}\NormalTok{possible_graded, first_attempted_timestamp), }
    \DataTypeTok{by =} \KeywordTok{c}\NormalTok{(}\StringTok{"hash_id"}\NormalTok{=}\StringTok{"hash_id"}\NormalTok{, }\StringTok{"module_id"}\NormalTok{=}\StringTok{"usage_key"}\NormalTok{)}
\NormalTok{)}

\CommentTok{# Only sequential modules have a grade associated with them}
\KeywordTok{table}\NormalTok{(ma}\OperatorTok{$}\NormalTok{module_type, ma}\OperatorTok{$}\NormalTok{first_attempted_timestamp}\OperatorTok{>}\DecValTok{0}\NormalTok{)}
\end{Highlighting}
\end{Shaded}

\begin{verbatim}
##                 
##                  TRUE
##   chapter           0
##   course            0
##   openassessment    0
##   problem           0
##   sequential     1163
##   video             0
\end{verbatim}

\begin{Shaded}
\begin{Highlighting}[]
\CommentTok{# We see that assignments were due (submitted) at different times}
\KeywordTok{boxplot}\NormalTok{(ma}\OperatorTok{$}\NormalTok{first_attempted_timestamp }\OperatorTok{~}\StringTok{ }\NormalTok{ma}\OperatorTok{$}\NormalTok{module_id)}
\end{Highlighting}
\end{Shaded}

\includegraphics{info4100.project_files/figure-latex/unnamed-chunk-3-1.pdf}

\hypertarget{step-2-define-a-prediction-task}{%
\section{Step 2: Define a prediction
task}\label{step-2-define-a-prediction-task}}

Recall the guidelines for defining a good prediction problem covered in
the Handbook chapter on prediction. You are looking for something
actionable (an opportunity to intervene) and a situation that repeats
(so the prediction can be useful in the future). The tradeoff with the
dataset you have here is that on the one hand it is very relevant to you
but on the other hand it is relatively small. Still, the data is
fine-grained and sufficiently messy to give you a taste of LMS data
analysis.

The prediction problem for this project is to build a one-day early
warning system for missing a graded submission. Specifically,
\textbf{your goal is to predict one day before the submission deadline,
if a student will forget to submit an assignment}, so that the system
can send a reminder. As you may have noticed during the data exploration
phase above (if not, you should go back and examine this), there are
several graded submissions and some students missed one or more of them.
We define \textbf{missing a submission} as having an NA for
\texttt{first\_attempted\_timestamp} but of course only for those that
are past due.

\hypertarget{instructions}{%
\subsubsection{Instructions}\label{instructions}}

\begin{enumerate}
\def\labelenumi{\arabic{enumi}.}
\tightlist
\item
  Treat each graded assignment as a prediction task (thus there are x*n
  prediction opportunities where x = number of graded assignments and n
  = 31 students).
\item
  Create a dataset that has 1 row per student per graded assessment with
  the binary outcome (did they MISS it? yes/no) and several predictors
  (see next tip)
\item
  Predictors (i.e.~features) need to be engineered with data from
  \textbf{24hrs before each assignment is due}, which of course varies
  across assignments; that means you have much more information to
  predict later assignments than earlier ones
\item
  Once your dataset is ready, split it into a training and a test set
\item
  Train a prediction model on the training data; you can try out any of
  the ones we have covered in the prediction homework and Random Forest
\item
  Keep tuning your model choice, model parameters (if any), and feature
  engineering
\item
  Finally, test your prediction accuracy on the test set
\end{enumerate}

\hypertarget{step-3-getting-you-started}{%
\section{Step 3: Getting you started}\label{step-3-getting-you-started}}

\hypertarget{create-the-outcome-variable}{%
\subsection{Create the outcome
variable}\label{create-the-outcome-variable}}

\textbf{Identify the graded assessments and whether a student did NOT
submit}. Recall we want to have a \emph{warning} system, so the outcome
should be the negative action.

Get the outcome for each graded assignment. Figure out the deadline for
each and compute the timestamp for 24hrs prior to the deadline. You
probably want to use the \texttt{ma} dataset I created for you above.

The following table helps you see the various graded assignments to
consider. We keep only those where possible\_graded \textgreater{} 0.
\textbf{I define the deadline as the 90th percentile of submissions (you
may use this simplification).}

\begin{Shaded}
\begin{Highlighting}[]
\NormalTok{ma }\OperatorTok\StringTok{ }
\StringTok{    }\KeywordTok{filter}\NormalTok{(possible_graded }\OperatorTok{>}\StringTok{ }\DecValTok{0}\NormalTok{) }\OperatorTok
\StringTok{    }\KeywordTok{group_by}\NormalTok{(module_id) }\OperatorTok\StringTok{ }
\StringTok{    }\KeywordTok{summarise}\NormalTok{(}
        \DataTypeTok{deadline =} \KeywordTok{quantile}\NormalTok{(first_attempted_timestamp, }\DataTypeTok{probs =} \FloatTok{.9}\NormalTok{, }\DataTypeTok{na.rm=}\NormalTok{T),}
        \DataTypeTok{p_unsubmitted =} \KeywordTok{mean}\NormalTok{(}\KeywordTok{is.na}\NormalTok{(first_attempted_timestamp))}
\NormalTok{    ) }\OperatorTok\StringTok{ }
\StringTok{    }\KeywordTok{arrange}\NormalTok{(deadline)}
\end{Highlighting}
\end{Shaded}

\begin{verbatim}
## `summarise()` ungrouping output (override with `.groups` argument)
\end{verbatim}

\begin{verbatim}
## # A tibble: 17 x 3
##    module_id                                              deadline p_unsubmitted
##    <chr>                                                     <dbl>         <dbl>
##  1 block-v1:Cornellx+INFO4100+Fall2020+type@sequential+~    1.60e9        0.0638
##  2 block-v1:Cornellx+INFO4100+Fall2020+type@sequential+~    1.60e9        0.0215
##  3 block-v1:Cornellx+INFO4100+Fall2020+type@sequential+~    1.60e9        0.0753
##  4 block-v1:Cornellx+INFO4100+Fall2020+type@sequential+~    1.60e9        0     
##  5 block-v1:Cornellx+INFO4100+Fall2020+type@sequential+~    1.60e9        0.0526
##  6 block-v1:Cornellx+INFO4100+Fall2020+type@sequential+~    1.60e9        0.0211
##  7 block-v1:Cornellx+INFO4100+Fall2020+type@sequential+~    1.60e9        0.0957
##  8 block-v1:Cornellx+INFO4100+Fall2020+type@sequential+~    1.60e9        0.0426
##  9 block-v1:Cornellx+INFO4100+Fall2020+type@sequential+~    1.60e9        0.0753
## 10 block-v1:Cornellx+INFO4100+Fall2020+type@sequential+~    1.60e9        0.0737
## 11 block-v1:Cornellx+INFO4100+Fall2020+type@sequential+~    1.60e9        0.0326
## 12 block-v1:Cornellx+INFO4100+Fall2020+type@sequential+~    1.60e9        0.0761
## 13 block-v1:Cornellx+INFO4100+Fall2020+type@sequential+~    1.60e9        0.0426
## 14 block-v1:Cornellx+INFO4100+Fall2020+type@sequential+~    1.60e9        0.571 
## 15 block-v1:Cornellx+INFO4100+Fall2020+type@sequential+~    1.60e9        0.909 
## 16 block-v1:Cornellx+INFO4100+Fall2020+type@sequential+~   NA             1     
## 17 block-v1:Cornellx+INFO4100+Fall2020+type@sequential+~   NA             1
\end{verbatim}

\begin{Shaded}
\begin{Highlighting}[]
\NormalTok{ma }\OperatorTok\StringTok{ }
\StringTok{    }\KeywordTok{filter}\NormalTok{((possible_graded }\OperatorTok{>}\StringTok{ }\DecValTok{0}\NormalTok{) }\OperatorTok{&}\NormalTok{(}\OperatorTok{!}\KeywordTok{is.na}\NormalTok{(first_attempted_timestamp))) }\OperatorTok
\StringTok{    }\KeywordTok{group_by}\NormalTok{(module_id) }\OperatorTok\StringTok{ }
\StringTok{    }\KeywordTok{summarise}\NormalTok{(}
        \DataTypeTok{deadline =} \KeywordTok{quantile}\NormalTok{(first_attempted_timestamp, }\DataTypeTok{probs =} \FloatTok{.9}\NormalTok{, }\DataTypeTok{na.rm=}\NormalTok{T),}
        \DataTypeTok{p_unsubmitted =} \KeywordTok{sum}\NormalTok{(first_attempted_timestamp}\OperatorTok{<}\NormalTok{deadline}\DecValTok{-60} \OperatorTok{*}\StringTok{ }\DecValTok{60} \OperatorTok{*}\StringTok{ }\DecValTok{24}\NormalTok{)}\OperatorTok{/}\KeywordTok{n}\NormalTok{()}
\NormalTok{    ) }\OperatorTok\StringTok{ }
\StringTok{    }\KeywordTok{arrange}\NormalTok{(deadline)}
\end{Highlighting}
\end{Shaded}

\begin{verbatim}
## `summarise()` ungrouping output (override with `.groups` argument)
\end{verbatim}

\begin{verbatim}
## # A tibble: 15 x 3
##    module_id                                              deadline p_unsubmitted
##    <chr>                                                     <dbl>         <dbl>
##  1 block-v1:Cornellx+INFO4100+Fall2020+type@sequential+~    1.60e9         0.841
##  2 block-v1:Cornellx+INFO4100+Fall2020+type@sequential+~    1.60e9         0.725
##  3 block-v1:Cornellx+INFO4100+Fall2020+type@sequential+~    1.60e9         0.709
##  4 block-v1:Cornellx+INFO4100+Fall2020+type@sequential+~    1.60e9         0.716
##  5 block-v1:Cornellx+INFO4100+Fall2020+type@sequential+~    1.60e9         0.656
##  6 block-v1:Cornellx+INFO4100+Fall2020+type@sequential+~    1.60e9         0.656
##  7 block-v1:Cornellx+INFO4100+Fall2020+type@sequential+~    1.60e9         0.682
##  8 block-v1:Cornellx+INFO4100+Fall2020+type@sequential+~    1.60e9         0.622
##  9 block-v1:Cornellx+INFO4100+Fall2020+type@sequential+~    1.60e9         0.593
## 10 block-v1:Cornellx+INFO4100+Fall2020+type@sequential+~    1.60e9         0.352
## 11 block-v1:Cornellx+INFO4100+Fall2020+type@sequential+~    1.60e9         0.607
## 12 block-v1:Cornellx+INFO4100+Fall2020+type@sequential+~    1.60e9         0.576
## 13 block-v1:Cornellx+INFO4100+Fall2020+type@sequential+~    1.60e9         0.644
## 14 block-v1:Cornellx+INFO4100+Fall2020+type@sequential+~    1.60e9         0    
## 15 block-v1:Cornellx+INFO4100+Fall2020+type@sequential+~    1.60e9         0
\end{verbatim}

Now you know which assessments (module\_ids) to target. \textbf{Be sure
to kick out the one with p\_unsubmitted \textgreater{} 0.5}; They were
not due yet when the export was created.

\emph{Question 2:} Now build a dataset with an indicator for each person
and each of these module\_ids with 1=unsubmitted, 0=submitted. Keep
track of the deadline: you only want to use features based on data up to
24hrs before it (i.e.~\texttt{24\ *\ 60\ *\ 60} seconds).

\begin{Shaded}
\begin{Highlighting}[]
\CommentTok{############################################### }
\CommentTok{####### }\RegionMarkerTok{BEGIN}\CommentTok{ INPUT: Define outcome ###########}
\CommentTok{###############################################}

\CommentTok{# edit ma to include a column with the module deadline}
\CommentTok{# and a binary column for the probability for whether the student submits on time}
\NormalTok{ma_edited =}\StringTok{ }\NormalTok{ma }\OperatorTok\StringTok{ }
\StringTok{    }\KeywordTok{filter}\NormalTok{(possible_graded }\OperatorTok{>}\StringTok{ }\DecValTok{0}\NormalTok{) }\OperatorTok
\StringTok{    }\KeywordTok{group_by}\NormalTok{(module_id) }\OperatorTok\StringTok{ }
\StringTok{    }\KeywordTok{summarise}\NormalTok{(}
        \DataTypeTok{deadline =} \KeywordTok{quantile}\NormalTok{(first_attempted_timestamp, }\DataTypeTok{probs =} \FloatTok{.9}\NormalTok{, }\DataTypeTok{na.rm=}\NormalTok{T),}
        \DataTypeTok{p_unsubmitted =} \KeywordTok{mean}\NormalTok{(}\KeywordTok{is.na}\NormalTok{(first_attempted_timestamp))}
\NormalTok{    ) }\OperatorTok\StringTok{ }
\StringTok{    }\KeywordTok{arrange}\NormalTok{(deadline) }\OperatorTok\StringTok{ }
\StringTok{    }\KeywordTok{filter}\NormalTok{(p_unsubmitted }\OperatorTok{<=}\FloatTok{0.5}\NormalTok{)}
\end{Highlighting}
\end{Shaded}

\begin{verbatim}
## `summarise()` ungrouping output (override with `.groups` argument)
\end{verbatim}

\begin{Shaded}
\begin{Highlighting}[]
\KeywordTok{print}\NormalTok{(ma_edited)}
\end{Highlighting}
\end{Shaded}

\begin{verbatim}
## # A tibble: 13 x 3
##    module_id                                              deadline p_unsubmitted
##    <chr>                                                     <dbl>         <dbl>
##  1 block-v1:Cornellx+INFO4100+Fall2020+type@sequential+~    1.60e9        0.0638
##  2 block-v1:Cornellx+INFO4100+Fall2020+type@sequential+~    1.60e9        0.0215
##  3 block-v1:Cornellx+INFO4100+Fall2020+type@sequential+~    1.60e9        0.0753
##  4 block-v1:Cornellx+INFO4100+Fall2020+type@sequential+~    1.60e9        0     
##  5 block-v1:Cornellx+INFO4100+Fall2020+type@sequential+~    1.60e9        0.0526
##  6 block-v1:Cornellx+INFO4100+Fall2020+type@sequential+~    1.60e9        0.0211
##  7 block-v1:Cornellx+INFO4100+Fall2020+type@sequential+~    1.60e9        0.0957
##  8 block-v1:Cornellx+INFO4100+Fall2020+type@sequential+~    1.60e9        0.0426
##  9 block-v1:Cornellx+INFO4100+Fall2020+type@sequential+~    1.60e9        0.0753
## 10 block-v1:Cornellx+INFO4100+Fall2020+type@sequential+~    1.60e9        0.0737
## 11 block-v1:Cornellx+INFO4100+Fall2020+type@sequential+~    1.60e9        0.0326
## 12 block-v1:Cornellx+INFO4100+Fall2020+type@sequential+~    1.60e9        0.0761
## 13 block-v1:Cornellx+INFO4100+Fall2020+type@sequential+~    1.60e9        0.0426
\end{verbatim}

\begin{Shaded}
\begin{Highlighting}[]
\CommentTok{# combine these new columns with ma }
\NormalTok{ma_combined =}\StringTok{ }\NormalTok{ma }\OperatorTok\StringTok{ }\KeywordTok{left_join}\NormalTok{(}
\NormalTok{    ma_edited }\OperatorTok\StringTok{ }\KeywordTok{select}\NormalTok{(module_id}\OperatorTok{:}\NormalTok{deadline,p_unsubmitted), }
    \DataTypeTok{by =} \KeywordTok{c}\NormalTok{(}\StringTok{"module_id"}\NormalTok{=}\StringTok{"module_id"}\NormalTok{)}
\NormalTok{)}

\CommentTok{# create outcome dataset}
\NormalTok{outcome =}\StringTok{ }\NormalTok{ma_combined }\OperatorTok\StringTok{ }
\StringTok{    }\KeywordTok{filter}\NormalTok{((possible_graded }\OperatorTok{>}\StringTok{ }\DecValTok{0}\NormalTok{)}\OperatorTok{&}\NormalTok{(p_unsubmitted }\OperatorTok{<=}\FloatTok{0.5}\NormalTok{)) }\OperatorTok
\StringTok{    }\KeywordTok{group_by}\NormalTok{(hash_id, module_id) }\OperatorTok\StringTok{ }
\StringTok{    }\KeywordTok{summarise}\NormalTok{(}
        \DataTypeTok{unsubmitted =}\NormalTok{ (}\KeywordTok{is.na}\NormalTok{(first_attempted_timestamp))}\OperatorTok{*}\DecValTok{1}\NormalTok{)}
\end{Highlighting}
\end{Shaded}

\begin{verbatim}
## `summarise()` regrouping output by 'hash_id' (override with `.groups` argument)
\end{verbatim}

\begin{Shaded}
\begin{Highlighting}[]
\CommentTok{# add deadline column from ma_edited}
\NormalTok{outcome =}\StringTok{ }\NormalTok{outcome }\OperatorTok\StringTok{ }\KeywordTok{left_join}\NormalTok{(}
\NormalTok{    ma_edited }\OperatorTok\StringTok{ }\KeywordTok{select}\NormalTok{(module_id}\OperatorTok{:}\NormalTok{deadline), }
    \DataTypeTok{by =} \KeywordTok{c}\NormalTok{(}\StringTok{"module_id"}\NormalTok{=}\StringTok{"module_id"}\NormalTok{)}
\NormalTok{  )}
\NormalTok{outcome}
\end{Highlighting}
\end{Shaded}

\begin{verbatim}
## # A tibble: 1,219 x 4
## # Groups:   hash_id [95]
##    hash_id            module_id                            unsubmitted  deadline
##    <chr>              <chr>                                      <dbl>     <dbl>
##  1 046b9f96c0155a63c~ block-v1:Cornellx+INFO4100+Fall2020~           0    1.60e9
##  2 046b9f96c0155a63c~ block-v1:Cornellx+INFO4100+Fall2020~           0    1.60e9
##  3 046b9f96c0155a63c~ block-v1:Cornellx+INFO4100+Fall2020~           0    1.60e9
##  4 046b9f96c0155a63c~ block-v1:Cornellx+INFO4100+Fall2020~           0    1.60e9
##  5 046b9f96c0155a63c~ block-v1:Cornellx+INFO4100+Fall2020~           0    1.60e9
##  6 046b9f96c0155a63c~ block-v1:Cornellx+INFO4100+Fall2020~           0    1.60e9
##  7 046b9f96c0155a63c~ block-v1:Cornellx+INFO4100+Fall2020~           0    1.60e9
##  8 046b9f96c0155a63c~ block-v1:Cornellx+INFO4100+Fall2020~           0    1.60e9
##  9 046b9f96c0155a63c~ block-v1:Cornellx+INFO4100+Fall2020~           0    1.60e9
## 10 046b9f96c0155a63c~ block-v1:Cornellx+INFO4100+Fall2020~           0    1.60e9
## # ... with 1,209 more rows
\end{verbatim}

\begin{Shaded}
\begin{Highlighting}[]
\CommentTok{############################################### }
\CommentTok{############################################### }
\end{Highlighting}
\end{Shaded}

\hypertarget{feature-engineering}{%
\subsection{Feature Engineering}\label{feature-engineering}}

\textbf{For each graded assessment, identify what data is appropriate
for feature engineering}

Before you start feature engineering, you need to constrain the data for
\textbf{each} assessment.

Remember that the dataset we are aiming for has 1 row per person and
assessment with several feature variables and one outcome variable. You
created the outcome above. Now you need to create the appropriate
features to join. I'm giving you an example for using
\texttt{deadline\ =\ 1600304996} and creating 2 basic features from the
clickstream. You should try to create a lot more features, including
complex ones, that can use the clickstream or other datasets (but
remember the timing constraint).

\begin{Shaded}
\begin{Highlighting}[]
\CommentTok{# PROF CODE EXAMPLE FEATURE}
\NormalTok{secs_day =}\StringTok{ }\DecValTok{60} \OperatorTok{*}\StringTok{ }\DecValTok{60} \OperatorTok{*}\StringTok{ }\DecValTok{24}
\NormalTok{example_deadline =}\StringTok{ }\DecValTok{1600304996}

\NormalTok{example_features =}\StringTok{ }\NormalTok{cl }\OperatorTok\StringTok{ }
\StringTok{    }\KeywordTok{filter}\NormalTok{(timestamp }\OperatorTok{<}\StringTok{ }\NormalTok{example_deadline }\OperatorTok{-}\StringTok{ }\NormalTok{secs_day) }\OperatorTok
\StringTok{    }\KeywordTok{group_by}\NormalTok{(hash_id) }\OperatorTok
\StringTok{    }\KeywordTok{summarise}\NormalTok{(}
        \DataTypeTok{num_events =} \KeywordTok{n}\NormalTok{(),}
        \DataTypeTok{num_seq_goto =} \KeywordTok{sum}\NormalTok{(event_type}\OperatorTok{==}\StringTok{"seq_goto"}\NormalTok{)}
\NormalTok{    )}
\end{Highlighting}
\end{Shaded}

\begin{verbatim}
## `summarise()` ungrouping output (override with `.groups` argument)
\end{verbatim}

\begin{Shaded}
\begin{Highlighting}[]
\KeywordTok{head}\NormalTok{(example_features)}
\end{Highlighting}
\end{Shaded}

\begin{verbatim}
## # A tibble: 6 x 3
##   hash_id                          num_events num_seq_goto
##   <chr>                                 <int>        <int>
## 1 046b9f96c0155a63c1262c8bd0a327e0        287           10
## 2 051123be58fb0985c3578ad867ec22e2        407            6
## 3 06b0110dadf9c128fd7550a2cd8a2bd4        190            2
## 4 08258abf7826f7c758ffe1ecfc468f17        695           27
## 5 0ca99849369b2ed808ee3060afe52eb6        829           64
## 6 0df790f3a775479a223f91ff3d888994        344            0
\end{verbatim}

\emph{Question 3:} Engineer features for each student and assessment,
subject to the timing constraint.

\begin{Shaded}
\begin{Highlighting}[]
\CommentTok{############################################### }
\CommentTok{###### }\RegionMarkerTok{BEGIN}\CommentTok{ INPUT: Engineer features #########}
\CommentTok{###############################################}

\CommentTok{# want dataset with 1 row per person and assessment, with several feature variables, and 1 outcome variable}
\NormalTok{features =}\StringTok{ }\NormalTok{outcome}
\NormalTok{ma_features =}\StringTok{ }\KeywordTok{data.frame}\NormalTok{()}
\NormalTok{cl_features =}\StringTok{ }\KeywordTok{data.frame}\NormalTok{()}
\CommentTok{# for each graded assessment, select data to use for feature eng based on module ddl-1 day}
\ControlFlowTok{for}\NormalTok{ (row }\ControlFlowTok{in} \DecValTok{1}\OperatorTok{:}\KeywordTok{nrow}\NormalTok{(ma_edited)) }\CommentTok{#nrow(ma_edited)==13}
\NormalTok{\{}
\NormalTok{  module_ddl =}\StringTok{ }\KeywordTok{as.numeric}\NormalTok{(ma_edited[row,}\StringTok{"deadline"}\NormalTok{])}
\NormalTok{  moduleid =}\StringTok{ }\KeywordTok{as.character}\NormalTok{(ma_edited[row,}\StringTok{"module_id"}\NormalTok{])}
  \CommentTok{### modules that are available before ddl + 1 day, including the current module}
  \CommentTok{#filter out assignments that weren't due yet and data from the window 24hours before the deadline}
\NormalTok{  example_features =}\StringTok{ }\NormalTok{ma_combined }\OperatorTok\StringTok{ }
\StringTok{    }\CommentTok{#filter((possible_graded >0 ) & (deadline <= module_ddl) & (first_attempted_timestamp < module_ddl - secs_day)) %>%}
\StringTok{    }\KeywordTok{filter}\NormalTok{((possible_graded }\OperatorTok{>}\DecValTok{0}\NormalTok{ ) }\OperatorTok{&}\StringTok{ }\NormalTok{(deadline }\OperatorTok{<=}\StringTok{ }\NormalTok{module_ddl }\OperatorTok{-}\StringTok{ }\NormalTok{secs_day)) }\OperatorTok
\StringTok{    }\KeywordTok{group_by}\NormalTok{(hash_id) }\OperatorTok
\StringTok{    }\KeywordTok{summarise}\NormalTok{(}
        \CommentTok{#add our features for assessments here!}
        \CommentTok{#num_assignments = n(),}
        \DataTypeTok{diff =} \KeywordTok{mean}\NormalTok{(deadline }\OperatorTok{+}\StringTok{ }\NormalTok{secs_day }\OperatorTok{-}\StringTok{ }\NormalTok{first_attempted_timestamp),}
        \CommentTok{#miss_ddl = sum(is.na(first_attempted_timestamp)),}
        \DataTypeTok{miss_ddl =} \KeywordTok{sum}\NormalTok{(deadline }\OperatorTok{-}\StringTok{ }\NormalTok{first_attempted_timestamp}\OperatorTok{<}\DecValTok{0}\NormalTok{),}
        \CommentTok{#avg_miss_ddl = sum(deadline - first_attempted_timestamp<0)/n(),}
        \DataTypeTok{avg_grade =} \KeywordTok{sum}\NormalTok{(earned_graded)}\OperatorTok{/}\KeywordTok{sum}\NormalTok{(possible_graded),}
        \CommentTok{#perf_score = sum(earned_graded==possible_graded),}
        \DataTypeTok{avg_perf_score =} \KeywordTok{sum}\NormalTok{(earned_graded}\OperatorTok{==}\NormalTok{possible_graded)}\OperatorTok{/}\KeywordTok{n}\NormalTok{(),}
        \CommentTok{#grade = earned_graded/possible_graded,}
        \CommentTok{#time_between_modification = mean(modified_timestamp-created_timestamp),}
        \CommentTok{# grade_above_avg = (earned_graded/possible_graded) > mean(earned_graded)/possible_graded,  # earned_graded/possible_graded) == mean(earned_graded)/possible_graded}
        \CommentTok{#more_than_1_possible_point = sum(possible_graded>1.0),}
        \CommentTok{#prev_attemp_last_hr = sum(first_attempted_timestamp > deadline - 3600), # not sure about this one}
        \CommentTok{#submitted_latest = !is.na(first_attempted_timestamp)}
\NormalTok{    )}
  \CommentTok{#feature checking if students submitted aldready the current assignment one day before the calculated deadline}
  
\NormalTok{  problem =}\StringTok{ }\NormalTok{ma_combined }\OperatorTok\StringTok{ }
\StringTok{    }\KeywordTok{filter}\NormalTok{((module_type}\OperatorTok{==}\StringTok{"problem"}\NormalTok{) }\OperatorTok{&}\StringTok{ }\NormalTok{(modified_timestamp }\OperatorTok{<=}\StringTok{ }\NormalTok{module_ddl }\OperatorTok{-}\StringTok{ }\NormalTok{secs_day))}\OperatorTok
\StringTok{    }\KeywordTok{group_by}\NormalTok{(hash_id) }\OperatorTok
\StringTok{    }\KeywordTok{summarise}\NormalTok{(}
      \CommentTok{#full_grade = sum(grade==max_grade),}
      \CommentTok{#time_spent = mean(modified_timestamp - created_timestamp)}
\NormalTok{    )}
\NormalTok{  video_info =}\StringTok{ }\NormalTok{ma_combined }\OperatorTok\StringTok{ }
\StringTok{    }\KeywordTok{filter}\NormalTok{((module_type}\OperatorTok{==}\StringTok{"video"}\NormalTok{) }\OperatorTok{&}\StringTok{ }\NormalTok{(modified_timestamp  }\OperatorTok{<=}\StringTok{ }\NormalTok{(module_ddl }\OperatorTok{-}\StringTok{ }\NormalTok{secs_day))) }\OperatorTok
\StringTok{    }\KeywordTok{group_by}\NormalTok{(hash_id) }\OperatorTok
\StringTok{    }\KeywordTok{summarise}\NormalTok{(}
        \CommentTok{#num_videos = n(),}
        \CommentTok{#time_on_videos = sum(modified_timestamp-created_timestamp),}
        \CommentTok{#time_per_video_avg = (sum(modified_timestamp-created_timestamp))/n()}
\NormalTok{    )}
  \CommentTok{#video_info}
\NormalTok{  features_per_module =}\StringTok{ }\NormalTok{features[}\KeywordTok{c}\NormalTok{(}\OperatorTok{-}\DecValTok{3}\NormalTok{,}\OperatorTok{-}\DecValTok{4}\NormalTok{)] }\OperatorTok\StringTok{ }\KeywordTok{filter}\NormalTok{(module_id }\OperatorTok{==}\StringTok{ }\NormalTok{moduleid) }\OperatorTok\StringTok{ }\KeywordTok{left_join}\NormalTok{(}
\NormalTok{    example_features, }
    \DataTypeTok{by =} \KeywordTok{c}\NormalTok{(}\StringTok{"hash_id"}\NormalTok{=}\StringTok{"hash_id"}\NormalTok{)}
\NormalTok{  )}
  \CommentTok{# features_per_module = features_per_module %>% left_join(}
  \CommentTok{#   video_info, }
  \CommentTok{#   by = c("hash_id"="hash_id")}
  \CommentTok{# )}
  \CommentTok{# features_per_module = features_per_module %>% left_join(}
  \CommentTok{#   problem, }
  \CommentTok{#   by = c("hash_id"="hash_id")}
  \CommentTok{# )}
  
\NormalTok{  submitted =}\StringTok{ }\NormalTok{ma_combined }\OperatorTok\StringTok{ }
\StringTok{    }\KeywordTok{filter}\NormalTok{((possible_graded }\OperatorTok{>}\DecValTok{0}\NormalTok{ ) }\OperatorTok{&}\StringTok{ }\NormalTok{(module_id }\OperatorTok{==}\StringTok{ }\NormalTok{moduleid)) }\OperatorTok
\StringTok{    }\KeywordTok{group_by}\NormalTok{(hash_id) }\OperatorTok
\StringTok{    }\KeywordTok{summarise}\NormalTok{(}
        \CommentTok{#add our features for assessments here!}
        \CommentTok{#num_assignments = n(),}
        \CommentTok{#diff = mean(deadline + secs_day - first_attempted_timestamp),}
        \DataTypeTok{not_submitted_by_ddl =}\NormalTok{ (}\OperatorTok{!}\NormalTok{(first_attempted_timestamp }\OperatorTok{<}\StringTok{ }\NormalTok{module_ddl }\OperatorTok{-}\StringTok{ }\NormalTok{secs_day))}\OperatorTok{*}\DecValTok{1}\NormalTok{,}
        \CommentTok{#miss_ddl = sum(deadline - first_attempted_timestamp<0),}
        \CommentTok{#avg_miss_ddl = sum(deadline - first_attempted_timestamp<0)/n(),}
        \CommentTok{#avg_grade = sum(earned_graded)/sum(possible_graded),}
        \CommentTok{#perf_score = sum(earned_graded==possible_graded),}
        \CommentTok{#avg_perf_score = sum(earned_graded==possible_graded)/n(),}
        \CommentTok{#grade = earned_graded/possible_graded,}
        \CommentTok{#time_between_modification = mean(modified_timestamp-created_timestamp),}
        \CommentTok{# grade_above_avg = (earned_graded/possible_graded) > mean(earned_graded)/possible_graded,  # earned_graded/possible_graded) == mean(earned_graded)/possible_graded}
        \CommentTok{#more_than_1_possible_point = sum(possible_graded>1.0),}
       \CommentTok{#  prev_attemp_last_hr = sum(first_attempted_timestamp > deadline - 3600), # not sure about this one}
\NormalTok{    )}
  \KeywordTok{print}\NormalTok{(submitted)}
\NormalTok{  features_per_module =}\StringTok{ }\NormalTok{features_per_module }\OperatorTok\StringTok{ }\KeywordTok{left_join}\NormalTok{(}
\NormalTok{    submitted,}
    \DataTypeTok{by =} \KeywordTok{c}\NormalTok{(}\StringTok{"hash_id"}\NormalTok{=}\StringTok{"hash_id"}\NormalTok{)}
\NormalTok{  )}
  
\NormalTok{  ma_features =}\StringTok{ }\KeywordTok{rbind}\NormalTok{(ma_features, features_per_module)}
  
  
\NormalTok{  cl_features_before_module_ddl =}\StringTok{ }\NormalTok{cl }\OperatorTok\StringTok{ }
\StringTok{    }\KeywordTok{filter}\NormalTok{(timestamp }\OperatorTok{<}\StringTok{ }\NormalTok{module_ddl }\OperatorTok{-}\StringTok{ }\NormalTok{secs_day) }\OperatorTok
\StringTok{    }\KeywordTok{group_by}\NormalTok{(hash_id) }\OperatorTok
\StringTok{    }\KeywordTok{summarise}\NormalTok{(}
      \CommentTok{# add our features for clickstream data here!}
        \DataTypeTok{num_events =} \KeywordTok{log2}\NormalTok{(}\KeywordTok{n}\NormalTok{()),}
        \CommentTok{#num_seq_goto = sum(event_type=="seq_goto"),}
        \CommentTok{#num_module = sum(str_detect(page,moduleid)),}
        \CommentTok{#pause_play_video = sum(event_type=="play_video"),}
        \CommentTok{#num_pause_video = sum(event_type=="pause_video"),}
        \DataTypeTok{num_video_interactions =} \KeywordTok{sum}\NormalTok{(}\KeywordTok{str_detect}\NormalTok{(event_type, }\StringTok{"video"}\NormalTok{)),}
        \CommentTok{#num_prob_check = sum(event_type=="problem_check"),}
        \CommentTok{#num_show_ans = sum(event_type=="showanswer"),}
        \DataTypeTok{num_links_clicked =} \KeywordTok{sum}\NormalTok{(name}\OperatorTok{==}\StringTok{"edx.ui.lms.link_clicked"}\NormalTok{),}
        \CommentTok{#num_ref_from_mail = sum(str_detect(referer, "mail")),}
        \CommentTok{#time_spent = mean(diff(timestamp)),}
        \DataTypeTok{num_events_past_week =} \KeywordTok{sum}\NormalTok{(timestamp }\OperatorTok{>}\StringTok{ }\NormalTok{module_ddl }\OperatorTok{-}\StringTok{ }\NormalTok{secs_day}\OperatorTok{*}\DecValTok{7}\NormalTok{),}
        \DataTypeTok{num_events_past_three_day=} \KeywordTok{sum}\NormalTok{(timestamp }\OperatorTok{>}\StringTok{ }\NormalTok{module_ddl }\OperatorTok{-}\StringTok{ }\NormalTok{secs_day}\OperatorTok{*}\DecValTok{4}\NormalTok{),}
        \DataTypeTok{num_events_past_day =} \KeywordTok{sum}\NormalTok{(timestamp }\OperatorTok{>}\StringTok{ }\NormalTok{module_ddl }\OperatorTok{-}\StringTok{ }\NormalTok{secs_day}\OperatorTok{*}\DecValTok{2}\NormalTok{),}
        \CommentTok{# add more event types (follow URL to figure out what they are)}
        \CommentTok{# use page}
        \CommentTok{# use name}
\NormalTok{    )}
  \CommentTok{#print(cl_features_before_module_ddl)}
\NormalTok{  cl_features_per_module =}\StringTok{ }\NormalTok{features[}\KeywordTok{c}\NormalTok{(}\OperatorTok{-}\DecValTok{3}\NormalTok{,}\OperatorTok{-}\DecValTok{4}\NormalTok{)] }\OperatorTok\StringTok{ }\KeywordTok{filter}\NormalTok{(module_id }\OperatorTok{==}\StringTok{ }\NormalTok{moduleid) }\OperatorTok\StringTok{ }\KeywordTok{left_join}\NormalTok{(}
\NormalTok{    cl_features_before_module_ddl, }
    \DataTypeTok{by =} \KeywordTok{c}\NormalTok{(}\StringTok{"hash_id"}\NormalTok{=}\StringTok{"hash_id"}\NormalTok{)}
\NormalTok{  )}
  \CommentTok{#print(features_per_module)}
\NormalTok{  cl_features =}\StringTok{ }\KeywordTok{rbind}\NormalTok{(cl_features, cl_features_per_module)}
\NormalTok{\}  }
\end{Highlighting}
\end{Shaded}

\begin{verbatim}
## `summarise()` ungrouping output (override with `.groups` argument)
## `summarise()` ungrouping output (override with `.groups` argument)
## `summarise()` ungrouping output (override with `.groups` argument)
## `summarise()` ungrouping output (override with `.groups` argument)
\end{verbatim}

\begin{verbatim}
## # A tibble: 94 x 2
##    hash_id                          not_submitted_by_ddl
##    <chr>                                           <dbl>
##  1 046b9f96c0155a63c1262c8bd0a327e0                    0
##  2 051123be58fb0985c3578ad867ec22e2                    0
##  3 06b0110dadf9c128fd7550a2cd8a2bd4                    0
##  4 08258abf7826f7c758ffe1ecfc468f17                    0
##  5 0ca99849369b2ed808ee3060afe52eb6                    0
##  6 0df790f3a775479a223f91ff3d888994                   NA
##  7 0e951c83667194cc73f4610350b72945                    0
##  8 0fefde5cc3b1c65dba071888ce9e522e                   NA
##  9 1403b390b640882382a0bb37983933fe                    0
## 10 14809cc5e83174cc0cb9d414bbb4fc9f                    0
## # ... with 84 more rows
\end{verbatim}

\begin{verbatim}
## `summarise()` ungrouping output (override with `.groups` argument)
## `summarise()` ungrouping output (override with `.groups` argument)
## `summarise()` ungrouping output (override with `.groups` argument)
## `summarise()` ungrouping output (override with `.groups` argument)
## `summarise()` ungrouping output (override with `.groups` argument)
\end{verbatim}

\begin{verbatim}
## # A tibble: 93 x 2
##    hash_id                          not_submitted_by_ddl
##    <chr>                                           <dbl>
##  1 046b9f96c0155a63c1262c8bd0a327e0                    0
##  2 051123be58fb0985c3578ad867ec22e2                    0
##  3 06b0110dadf9c128fd7550a2cd8a2bd4                    1
##  4 08258abf7826f7c758ffe1ecfc468f17                    0
##  5 0ca99849369b2ed808ee3060afe52eb6                    0
##  6 0df790f3a775479a223f91ff3d888994                    1
##  7 0e951c83667194cc73f4610350b72945                    0
##  8 0fefde5cc3b1c65dba071888ce9e522e                    1
##  9 1403b390b640882382a0bb37983933fe                    0
## 10 14809cc5e83174cc0cb9d414bbb4fc9f                    0
## # ... with 83 more rows
\end{verbatim}

\begin{verbatim}
## `summarise()` ungrouping output (override with `.groups` argument)
## `summarise()` ungrouping output (override with `.groups` argument)
## `summarise()` ungrouping output (override with `.groups` argument)
## `summarise()` ungrouping output (override with `.groups` argument)
## `summarise()` ungrouping output (override with `.groups` argument)
\end{verbatim}

\begin{verbatim}
## # A tibble: 93 x 2
##    hash_id                          not_submitted_by_ddl
##    <chr>                                           <dbl>
##  1 046b9f96c0155a63c1262c8bd0a327e0                    0
##  2 051123be58fb0985c3578ad867ec22e2                    0
##  3 06b0110dadf9c128fd7550a2cd8a2bd4                    1
##  4 08258abf7826f7c758ffe1ecfc468f17                    0
##  5 0ca99849369b2ed808ee3060afe52eb6                    0
##  6 0df790f3a775479a223f91ff3d888994                   NA
##  7 0e951c83667194cc73f4610350b72945                    0
##  8 0fefde5cc3b1c65dba071888ce9e522e                   NA
##  9 1403b390b640882382a0bb37983933fe                    0
## 10 14809cc5e83174cc0cb9d414bbb4fc9f                    0
## # ... with 83 more rows
\end{verbatim}

\begin{verbatim}
## `summarise()` ungrouping output (override with `.groups` argument)
## `summarise()` ungrouping output (override with `.groups` argument)
## `summarise()` ungrouping output (override with `.groups` argument)
## `summarise()` ungrouping output (override with `.groups` argument)
## `summarise()` ungrouping output (override with `.groups` argument)
\end{verbatim}

\begin{verbatim}
## # A tibble: 95 x 2
##    hash_id                          not_submitted_by_ddl
##    <chr>                                           <dbl>
##  1 046b9f96c0155a63c1262c8bd0a327e0                    1
##  2 051123be58fb0985c3578ad867ec22e2                    0
##  3 06b0110dadf9c128fd7550a2cd8a2bd4                    1
##  4 08258abf7826f7c758ffe1ecfc468f17                    0
##  5 0ca99849369b2ed808ee3060afe52eb6                    0
##  6 0df790f3a775479a223f91ff3d888994                    1
##  7 0e951c83667194cc73f4610350b72945                    0
##  8 0fefde5cc3b1c65dba071888ce9e522e                    0
##  9 1403b390b640882382a0bb37983933fe                    0
## 10 14809cc5e83174cc0cb9d414bbb4fc9f                    0
## # ... with 85 more rows
\end{verbatim}

\begin{verbatim}
## `summarise()` ungrouping output (override with `.groups` argument)
## `summarise()` ungrouping output (override with `.groups` argument)
## `summarise()` ungrouping output (override with `.groups` argument)
## `summarise()` ungrouping output (override with `.groups` argument)
## `summarise()` ungrouping output (override with `.groups` argument)
\end{verbatim}

\begin{verbatim}
## # A tibble: 95 x 2
##    hash_id                          not_submitted_by_ddl
##    <chr>                                           <dbl>
##  1 046b9f96c0155a63c1262c8bd0a327e0                    1
##  2 051123be58fb0985c3578ad867ec22e2                    0
##  3 06b0110dadf9c128fd7550a2cd8a2bd4                    1
##  4 08258abf7826f7c758ffe1ecfc468f17                    0
##  5 0ca99849369b2ed808ee3060afe52eb6                    0
##  6 0df790f3a775479a223f91ff3d888994                    1
##  7 0e951c83667194cc73f4610350b72945                   NA
##  8 0fefde5cc3b1c65dba071888ce9e522e                    0
##  9 1403b390b640882382a0bb37983933fe                    0
## 10 14809cc5e83174cc0cb9d414bbb4fc9f                    0
## # ... with 85 more rows
\end{verbatim}

\begin{verbatim}
## `summarise()` ungrouping output (override with `.groups` argument)
## `summarise()` ungrouping output (override with `.groups` argument)
## `summarise()` ungrouping output (override with `.groups` argument)
## `summarise()` ungrouping output (override with `.groups` argument)
## `summarise()` ungrouping output (override with `.groups` argument)
\end{verbatim}

\begin{verbatim}
## # A tibble: 95 x 2
##    hash_id                          not_submitted_by_ddl
##    <chr>                                           <dbl>
##  1 046b9f96c0155a63c1262c8bd0a327e0                    1
##  2 051123be58fb0985c3578ad867ec22e2                    1
##  3 06b0110dadf9c128fd7550a2cd8a2bd4                    1
##  4 08258abf7826f7c758ffe1ecfc468f17                    0
##  5 0ca99849369b2ed808ee3060afe52eb6                    0
##  6 0df790f3a775479a223f91ff3d888994                    0
##  7 0e951c83667194cc73f4610350b72945                    0
##  8 0fefde5cc3b1c65dba071888ce9e522e                    1
##  9 1403b390b640882382a0bb37983933fe                    0
## 10 14809cc5e83174cc0cb9d414bbb4fc9f                    0
## # ... with 85 more rows
\end{verbatim}

\begin{verbatim}
## `summarise()` ungrouping output (override with `.groups` argument)
## `summarise()` ungrouping output (override with `.groups` argument)
## `summarise()` ungrouping output (override with `.groups` argument)
## `summarise()` ungrouping output (override with `.groups` argument)
## `summarise()` ungrouping output (override with `.groups` argument)
\end{verbatim}

\begin{verbatim}
## # A tibble: 94 x 2
##    hash_id                          not_submitted_by_ddl
##    <chr>                                           <dbl>
##  1 046b9f96c0155a63c1262c8bd0a327e0                    0
##  2 051123be58fb0985c3578ad867ec22e2                    0
##  3 06b0110dadf9c128fd7550a2cd8a2bd4                    1
##  4 08258abf7826f7c758ffe1ecfc468f17                    0
##  5 0ca99849369b2ed808ee3060afe52eb6                    0
##  6 0df790f3a775479a223f91ff3d888994                    1
##  7 0e951c83667194cc73f4610350b72945                    0
##  8 0fefde5cc3b1c65dba071888ce9e522e                    0
##  9 1403b390b640882382a0bb37983933fe                    0
## 10 14809cc5e83174cc0cb9d414bbb4fc9f                    0
## # ... with 84 more rows
\end{verbatim}

\begin{verbatim}
## `summarise()` ungrouping output (override with `.groups` argument)
## `summarise()` ungrouping output (override with `.groups` argument)
## `summarise()` ungrouping output (override with `.groups` argument)
## `summarise()` ungrouping output (override with `.groups` argument)
## `summarise()` ungrouping output (override with `.groups` argument)
\end{verbatim}

\begin{verbatim}
## # A tibble: 94 x 2
##    hash_id                          not_submitted_by_ddl
##    <chr>                                           <dbl>
##  1 046b9f96c0155a63c1262c8bd0a327e0                    1
##  2 051123be58fb0985c3578ad867ec22e2                    1
##  3 06b0110dadf9c128fd7550a2cd8a2bd4                    1
##  4 08258abf7826f7c758ffe1ecfc468f17                    1
##  5 0ca99849369b2ed808ee3060afe52eb6                    0
##  6 0df790f3a775479a223f91ff3d888994                    0
##  7 0e951c83667194cc73f4610350b72945                    0
##  8 0fefde5cc3b1c65dba071888ce9e522e                    1
##  9 1403b390b640882382a0bb37983933fe                    1
## 10 14809cc5e83174cc0cb9d414bbb4fc9f                    0
## # ... with 84 more rows
\end{verbatim}

\begin{verbatim}
## `summarise()` ungrouping output (override with `.groups` argument)
## `summarise()` ungrouping output (override with `.groups` argument)
## `summarise()` ungrouping output (override with `.groups` argument)
## `summarise()` ungrouping output (override with `.groups` argument)
## `summarise()` ungrouping output (override with `.groups` argument)
\end{verbatim}

\begin{verbatim}
## # A tibble: 93 x 2
##    hash_id                          not_submitted_by_ddl
##    <chr>                                           <dbl>
##  1 046b9f96c0155a63c1262c8bd0a327e0                    1
##  2 051123be58fb0985c3578ad867ec22e2                    1
##  3 06b0110dadf9c128fd7550a2cd8a2bd4                    1
##  4 08258abf7826f7c758ffe1ecfc468f17                    1
##  5 0ca99849369b2ed808ee3060afe52eb6                    0
##  6 0df790f3a775479a223f91ff3d888994                    0
##  7 0e951c83667194cc73f4610350b72945                    0
##  8 0fefde5cc3b1c65dba071888ce9e522e                    1
##  9 1403b390b640882382a0bb37983933fe                   NA
## 10 14809cc5e83174cc0cb9d414bbb4fc9f                    0
## # ... with 83 more rows
\end{verbatim}

\begin{verbatim}
## `summarise()` ungrouping output (override with `.groups` argument)
## `summarise()` ungrouping output (override with `.groups` argument)
## `summarise()` ungrouping output (override with `.groups` argument)
## `summarise()` ungrouping output (override with `.groups` argument)
## `summarise()` ungrouping output (override with `.groups` argument)
\end{verbatim}

\begin{verbatim}
## # A tibble: 95 x 2
##    hash_id                          not_submitted_by_ddl
##    <chr>                                           <dbl>
##  1 046b9f96c0155a63c1262c8bd0a327e0                    1
##  2 051123be58fb0985c3578ad867ec22e2                    0
##  3 06b0110dadf9c128fd7550a2cd8a2bd4                    1
##  4 08258abf7826f7c758ffe1ecfc468f17                    1
##  5 0ca99849369b2ed808ee3060afe52eb6                    1
##  6 0df790f3a775479a223f91ff3d888994                   NA
##  7 0e951c83667194cc73f4610350b72945                    1
##  8 0fefde5cc3b1c65dba071888ce9e522e                    1
##  9 1403b390b640882382a0bb37983933fe                    0
## 10 14809cc5e83174cc0cb9d414bbb4fc9f                   NA
## # ... with 85 more rows
\end{verbatim}

\begin{verbatim}
## `summarise()` ungrouping output (override with `.groups` argument)
## `summarise()` ungrouping output (override with `.groups` argument)
## `summarise()` ungrouping output (override with `.groups` argument)
## `summarise()` ungrouping output (override with `.groups` argument)
## `summarise()` ungrouping output (override with `.groups` argument)
\end{verbatim}

\begin{verbatim}
## # A tibble: 92 x 2
##    hash_id                          not_submitted_by_ddl
##    <chr>                                           <dbl>
##  1 046b9f96c0155a63c1262c8bd0a327e0                    1
##  2 051123be58fb0985c3578ad867ec22e2                    1
##  3 06b0110dadf9c128fd7550a2cd8a2bd4                    1
##  4 08258abf7826f7c758ffe1ecfc468f17                    1
##  5 0ca99849369b2ed808ee3060afe52eb6                    0
##  6 0df790f3a775479a223f91ff3d888994                    1
##  7 0e951c83667194cc73f4610350b72945                    0
##  8 0fefde5cc3b1c65dba071888ce9e522e                    1
##  9 1403b390b640882382a0bb37983933fe                    0
## 10 14809cc5e83174cc0cb9d414bbb4fc9f                    0
## # ... with 82 more rows
\end{verbatim}

\begin{verbatim}
## `summarise()` ungrouping output (override with `.groups` argument)
## `summarise()` ungrouping output (override with `.groups` argument)
## `summarise()` ungrouping output (override with `.groups` argument)
## `summarise()` ungrouping output (override with `.groups` argument)
## `summarise()` ungrouping output (override with `.groups` argument)
\end{verbatim}

\begin{verbatim}
## # A tibble: 92 x 2
##    hash_id                          not_submitted_by_ddl
##    <chr>                                           <dbl>
##  1 046b9f96c0155a63c1262c8bd0a327e0                    1
##  2 051123be58fb0985c3578ad867ec22e2                    1
##  3 06b0110dadf9c128fd7550a2cd8a2bd4                   NA
##  4 08258abf7826f7c758ffe1ecfc468f17                    1
##  5 0ca99849369b2ed808ee3060afe52eb6                    0
##  6 0df790f3a775479a223f91ff3d888994                    1
##  7 0e951c83667194cc73f4610350b72945                    0
##  8 0fefde5cc3b1c65dba071888ce9e522e                    1
##  9 1403b390b640882382a0bb37983933fe                    0
## 10 14809cc5e83174cc0cb9d414bbb4fc9f                    0
## # ... with 82 more rows
\end{verbatim}

\begin{verbatim}
## `summarise()` ungrouping output (override with `.groups` argument)
## `summarise()` ungrouping output (override with `.groups` argument)
## `summarise()` ungrouping output (override with `.groups` argument)
## `summarise()` ungrouping output (override with `.groups` argument)
## `summarise()` ungrouping output (override with `.groups` argument)
\end{verbatim}

\begin{verbatim}
## # A tibble: 94 x 2
##    hash_id                          not_submitted_by_ddl
##    <chr>                                           <dbl>
##  1 046b9f96c0155a63c1262c8bd0a327e0                    1
##  2 051123be58fb0985c3578ad867ec22e2                    1
##  3 06b0110dadf9c128fd7550a2cd8a2bd4                    1
##  4 08258abf7826f7c758ffe1ecfc468f17                    0
##  5 0ca99849369b2ed808ee3060afe52eb6                    0
##  6 0df790f3a775479a223f91ff3d888994                    0
##  7 0e951c83667194cc73f4610350b72945                    0
##  8 0fefde5cc3b1c65dba071888ce9e522e                    0
##  9 1403b390b640882382a0bb37983933fe                    0
## 10 14809cc5e83174cc0cb9d414bbb4fc9f                    0
## # ... with 84 more rows
\end{verbatim}

\begin{verbatim}
## `summarise()` ungrouping output (override with `.groups` argument)
\end{verbatim}

\begin{Shaded}
\begin{Highlighting}[]
\CommentTok{#print(cl_features)}

\CommentTok{# how do we save each set of features for each module into one dataset? }
\CommentTok{#print(ma_features)}
\NormalTok{all_features =}\StringTok{ }\KeywordTok{left_join}\NormalTok{(ma_features, cl_features, }\DataTypeTok{by =} \KeywordTok{c}\NormalTok{(}\StringTok{"hash_id"}\NormalTok{=}\StringTok{"hash_id"}\NormalTok{, }\StringTok{"module_id"}\NormalTok{=}\StringTok{"module_id"}\NormalTok{))}
\NormalTok{all_features}
\end{Highlighting}
\end{Shaded}

\begin{verbatim}
## # A tibble: 1,219 x 13
## # Groups:   hash_id [95]
##    hash_id module_id  diff miss_ddl avg_grade avg_perf_score not_submitted_b~
##    <chr>   <chr>     <dbl>    <int>     <dbl>          <dbl>            <dbl>
##  1 046b9f~ block-v1~    NA       NA        NA             NA                0
##  2 051123~ block-v1~    NA       NA        NA             NA                0
##  3 06b011~ block-v1~    NA       NA        NA             NA                0
##  4 08258a~ block-v1~    NA       NA        NA             NA                0
##  5 0ca998~ block-v1~    NA       NA        NA             NA                0
##  6 0df790~ block-v1~    NA       NA        NA             NA               NA
##  7 0e951c~ block-v1~    NA       NA        NA             NA                0
##  8 0fefde~ block-v1~    NA       NA        NA             NA               NA
##  9 1403b3~ block-v1~    NA       NA        NA             NA                0
## 10 14809c~ block-v1~    NA       NA        NA             NA                0
## # ... with 1,209 more rows, and 6 more variables: num_events <dbl>,
## #   num_video_interactions <int>, num_links_clicked <int>,
## #   num_events_past_week <int>, num_events_past_three_day <int>,
## #   num_events_past_day <int>
\end{verbatim}

\begin{Shaded}
\begin{Highlighting}[]
\CommentTok{###############################################}
\CommentTok{###############################################}
\end{Highlighting}
\end{Shaded}

\hypertarget{step-4-split-your-dataset}{%
\section{Step 4: Split your dataset}\label{step-4-split-your-dataset}}

\emph{Question 4:} We would like train the model on earlier assessments
in order to make early alert predictions for later ones. As the hold-out
test set, designate the four (4) last assessments (i.e.~with the 4
latest computed deadlines, or the last 4 periods; same thing). You will
use all the remaining data to train. Note that this may not be the best
setup for all applications (e.g.~if we wanted to use the model at the
start of the course next year, but it is a reasonable approach if we
wanted to use the model for the rest of this course offering). Identify
the module\_ids of the last four assignments, put data associated with
their periods in the \texttt{test} dataset. Take all the remaining data
(earlier periods excl the last 4) and put it in the \texttt{train}
dataset.

\begin{Shaded}
\begin{Highlighting}[]
\CommentTok{############################################### }
\CommentTok{######## }\RegionMarkerTok{BEGIN}\CommentTok{ INPUT: Split dataset ###########}
\CommentTok{###############################################}

\CommentTok{# Identify last 4 periods for testing}
\CommentTok{# add code here}
\NormalTok{all_features =}\StringTok{ }\KeywordTok{left_join}\NormalTok{(ma_features, cl_features, }\DataTypeTok{by =} \KeywordTok{c}\NormalTok{(}\StringTok{"hash_id"}\NormalTok{=}\StringTok{"hash_id"}\NormalTok{, }\StringTok{"module_id"}\NormalTok{=}\StringTok{"module_id"}\NormalTok{))}
\NormalTok{all_features =}\StringTok{ }\KeywordTok{left_join}\NormalTok{(all_features, outcome, }\DataTypeTok{by =} \KeywordTok{c}\NormalTok{(}\StringTok{"hash_id"}\NormalTok{=}\StringTok{"hash_id"}\NormalTok{, }\StringTok{"module_id"}\NormalTok{=}\StringTok{"module_id"}\NormalTok{))}
\KeywordTok{sum}\NormalTok{(}\KeywordTok{is.na}\NormalTok{(all_features[[}\StringTok{"unsubmitted"}\NormalTok{]])) }\CommentTok{#==0}
\end{Highlighting}
\end{Shaded}

\begin{verbatim}
## [1] 0
\end{verbatim}

\begin{Shaded}
\begin{Highlighting}[]
\KeywordTok{sum}\NormalTok{(}\KeywordTok{is.na}\NormalTok{(all_features[[}\StringTok{"miss_ddl"}\NormalTok{]])) }\CommentTok{# 96}
\end{Highlighting}
\end{Shaded}

\begin{verbatim}
## [1] 307
\end{verbatim}

\begin{Shaded}
\begin{Highlighting}[]
\KeywordTok{sum}\NormalTok{(}\KeywordTok{is.na}\NormalTok{(all_features[[}\StringTok{"num_events"}\NormalTok{]])) }\CommentTok{# 11}
\end{Highlighting}
\end{Shaded}

\begin{verbatim}
## [1] 11
\end{verbatim}

\begin{Shaded}
\begin{Highlighting}[]
\KeywordTok{sum}\NormalTok{(all_features[[}\StringTok{"unsubmitted"}\NormalTok{]]}\OperatorTok{==}\DecValTok{1}\NormalTok{) }\CommentTok{#63}
\end{Highlighting}
\end{Shaded}

\begin{verbatim}
## [1] 63
\end{verbatim}

\begin{Shaded}
\begin{Highlighting}[]
\CommentTok{#replace_na(1) will result in classification tree not working}
\ControlFlowTok{for}\NormalTok{ (col }\ControlFlowTok{in} \KeywordTok{colnames}\NormalTok{(all_features))\{}
  \CommentTok{#if (col=="miss_ddl" || col=="unsubmitted")\{}
  \CommentTok{#  all_features[[col]] = all_features[[col]] %>% replace_na(1)}
  \CommentTok{#\}}
  \CommentTok{#else\{}
\NormalTok{    all_features[[col]] =}\StringTok{ }\NormalTok{all_features[[col]] }\OperatorTok\StringTok{ }\KeywordTok{replace_na}\NormalTok{(}\DecValTok{0}\NormalTok{)}
  \CommentTok{#\}}
\NormalTok{\}}
\CommentTok{#all_features =  all_features %>% drop_na()}
\CommentTok{#all_features_dropped_na}
\CommentTok{#all_features = all_features %>% dplyr::mutate(x = replace_na(diff, 0))}
\CommentTok{#$diff = all_features$diff %>% replace_na(0)}
\CommentTok{#all_features$avg_perf_score = all_features$avg_perf_score %>% replace_na(0)}
\CommentTok{#all_features$num_events = all_features$num_events %>% replace_na(0)}
\NormalTok{all_features}
\end{Highlighting}
\end{Shaded}

\begin{verbatim}
## # A tibble: 1,219 x 15
## # Groups:   hash_id [95]
##    hash_id module_id  diff miss_ddl avg_grade avg_perf_score not_submitted_b~
##    <chr>   <chr>     <dbl>    <dbl>     <dbl>          <dbl>            <dbl>
##  1 046b9f~ block-v1~     0        0         0              0                0
##  2 051123~ block-v1~     0        0         0              0                0
##  3 06b011~ block-v1~     0        0         0              0                0
##  4 08258a~ block-v1~     0        0         0              0                0
##  5 0ca998~ block-v1~     0        0         0              0                0
##  6 0df790~ block-v1~     0        0         0              0                0
##  7 0e951c~ block-v1~     0        0         0              0                0
##  8 0fefde~ block-v1~     0        0         0              0                0
##  9 1403b3~ block-v1~     0        0         0              0                0
## 10 14809c~ block-v1~     0        0         0              0                0
## # ... with 1,209 more rows, and 8 more variables: num_events <dbl>,
## #   num_video_interactions <dbl>, num_links_clicked <dbl>,
## #   num_events_past_week <dbl>, num_events_past_three_day <dbl>,
## #   num_events_past_day <dbl>, unsubmitted <dbl>, deadline <dbl>
\end{verbatim}

\begin{Shaded}
\begin{Highlighting}[]
\NormalTok{last_four =}\StringTok{ }\NormalTok{ma_edited}\OperatorTok{$}\NormalTok{module_id[}\DecValTok{10}\OperatorTok{:}\DecValTok{13}\NormalTok{]}
\CommentTok{#print(last_four)}

\NormalTok{modelingTime =}\StringTok{ }\DecValTok{1602288653} \OperatorTok{+}\StringTok{ }\DecValTok{1000000}     
\NormalTok{K =}\StringTok{ }\DecValTok{1000000}
\NormalTok{T =}\StringTok{ }\DecValTok{1}\OperatorTok{/}\DecValTok{2}
\CommentTok{#all_features$weight = K*exp(-(modelingTime - all_features$deadline)/T)}
\NormalTok{all_features}\OperatorTok{$}\NormalTok{weight =}\StringTok{ }\DecValTok{1}\OperatorTok{/}\NormalTok{((modelingTime }\OperatorTok{-}\StringTok{ }\NormalTok{all_features}\OperatorTok{$}\NormalTok{deadline)}\OperatorTok{/}\NormalTok{K)}
\CommentTok{# Split the dataset into train and test based on the module_ids or periods}
\NormalTok{ test =}\StringTok{ }\NormalTok{all_features }\OperatorTok\StringTok{ }\KeywordTok{filter}\NormalTok{(module_id }\OperatorTok\StringTok{ }\NormalTok{last_four)}
 \KeywordTok{print}\NormalTok{(test)}
\end{Highlighting}
\end{Shaded}

\begin{verbatim}
## # A tibble: 373 x 16
## # Groups:   hash_id [95]
##    hash_id module_id    diff miss_ddl avg_grade avg_perf_score not_submitted_b~
##    <chr>   <chr>       <dbl>    <dbl>     <dbl>          <dbl>            <dbl>
##  1 046b9f~ block-v1~ 192892.        1     0.955          0.667                1
##  2 051123~ block-v1~ 238216.        0     0.857          0.778                0
##  3 06b011~ block-v1~  -3831.        5     0.904          0.556                1
##  4 08258a~ block-v1~ 263935.        0     0.915          0.667                1
##  5 0ca998~ block-v1~ 415542.        0     0.994          0.889                1
##  6 0df790~ block-v1~      0         0     0.644          0.556                0
##  7 0e951c~ block-v1~      0         0     0.960          0.778                1
##  8 0fefde~ block-v1~      0         0     0.782          0.556                1
##  9 1403b3~ block-v1~      0         0     0.915          0.556                0
## 10 14809c~ block-v1~ 353794.        0     0.891          0.778                0
## # ... with 363 more rows, and 9 more variables: num_events <dbl>,
## #   num_video_interactions <dbl>, num_links_clicked <dbl>,
## #   num_events_past_week <dbl>, num_events_past_three_day <dbl>,
## #   num_events_past_day <dbl>, unsubmitted <dbl>, deadline <dbl>, weight <dbl>
\end{verbatim}

\begin{Shaded}
\begin{Highlighting}[]
\NormalTok{ train =}\StringTok{ }\NormalTok{all_features }\OperatorTok\StringTok{ }\KeywordTok{filter}\NormalTok{(}\OperatorTok{!}\NormalTok{module_id }\OperatorTok\StringTok{ }\KeywordTok{c}\NormalTok{(last_four,}\StringTok{"block-v1:Cornellx+INFO4100+Fall2020+type@sequential+block@bc80acbecf2d43f7b1d24704ff03fdf3"}\NormalTok{))}
 \KeywordTok{print}\NormalTok{(train)}
\end{Highlighting}
\end{Shaded}

\begin{verbatim}
## # A tibble: 752 x 16
## # Groups:   hash_id [95]
##    hash_id module_id   diff miss_ddl avg_grade avg_perf_score not_submitted_b~
##    <chr>   <chr>      <dbl>    <dbl>     <dbl>          <dbl>            <dbl>
##  1 046b9f~ block-v1~ 3.67e5        0         1              1                0
##  2 051123~ block-v1~ 3.61e5        0         1              1                0
##  3 06b011~ block-v1~ 3.59e5        0         1              1                1
##  4 08258a~ block-v1~ 3.58e5        0         1              1                0
##  5 0ca998~ block-v1~ 3.63e5        0         1              1                0
##  6 0df790~ block-v1~ 0.            0         0              0                1
##  7 0e951c~ block-v1~ 2.41e5        0         1              1                0
##  8 0fefde~ block-v1~ 0.            0         0              0                1
##  9 1403b3~ block-v1~ 3.25e5        0         1              1                0
## 10 14809c~ block-v1~ 3.68e5        0         1              1                0
## # ... with 742 more rows, and 9 more variables: num_events <dbl>,
## #   num_video_interactions <dbl>, num_links_clicked <dbl>,
## #   num_events_past_week <dbl>, num_events_past_three_day <dbl>,
## #   num_events_past_day <dbl>, unsubmitted <dbl>, deadline <dbl>, weight <dbl>
\end{verbatim}

\begin{Shaded}
\begin{Highlighting}[]
\KeywordTok{summary}\NormalTok{(all_features)}
\end{Highlighting}
\end{Shaded}

\begin{verbatim}
##    hash_id           module_id              diff            miss_ddl     
##  Length:1219        Length:1219        Min.   :-858902   Min.   :0.0000  
##  Class :character   Class :character   1st Qu.:      0   1st Qu.:0.0000  
##  Mode  :character   Mode  :character   Median : 235871   Median :0.0000  
##                                        Mean   : 209399   Mean   :0.3314  
##                                        3rd Qu.: 352805   3rd Qu.:0.0000  
##                                        Max.   : 502947   Max.   :7.0000  
##    avg_grade      avg_perf_score   not_submitted_by_ddl   num_events    
##  Min.   :0.0000   Min.   :0.0000   Min.   :0.0000       Min.   : 0.000  
##  1st Qu.:0.9144   1st Qu.:0.7143   1st Qu.:0.0000       1st Qu.: 8.234  
##  Median :0.9796   Median :0.8571   Median :0.0000       Median : 9.236  
##  Mean   :0.8523   Mean   :0.7777   Mean   :0.3363       Mean   : 8.851  
##  3rd Qu.:1.0000   3rd Qu.:1.0000   3rd Qu.:1.0000       3rd Qu.: 9.870  
##  Max.   :1.0000   Max.   :1.0000   Max.   :1.0000       Max.   :11.185  
##  num_video_interactions num_links_clicked num_events_past_week
##  Min.   :  0.00         Min.   :  0.00    Min.   :   0.0      
##  1st Qu.: 32.00         1st Qu.: 23.00    1st Qu.: 101.0      
##  Median : 58.00         Median : 44.00    Median : 177.0      
##  Mean   : 87.44         Mean   : 50.06    Mean   : 188.2      
##  3rd Qu.:110.50         3rd Qu.: 69.00    3rd Qu.: 251.0      
##  Max.   :771.00         Max.   :183.00    Max.   :1026.0      
##  num_events_past_three_day num_events_past_day  unsubmitted     
##  Min.   :  0.0             Min.   :  0.00      Min.   :0.00000  
##  1st Qu.: 32.5             1st Qu.:  0.00      1st Qu.:0.00000  
##  Median : 80.0             Median : 15.00      Median :0.00000  
##  Mean   :100.6             Mean   : 35.57      Mean   :0.05168  
##  3rd Qu.:150.0             3rd Qu.: 48.00      3rd Qu.:0.00000  
##  Max.   :763.0             Max.   :698.00      Max.   :1.00000  
##     deadline             weight      
##  Min.   :1.599e+09   Min.   :0.2531  
##  1st Qu.:1.600e+09   1st Qu.:0.3352  
##  Median :1.601e+09   Median :0.4360  
##  Mean   :1.601e+09   Mean   :0.5149  
##  3rd Qu.:1.602e+09   3rd Qu.:0.6164  
##  Max.   :1.602e+09   Max.   :1.0000
\end{verbatim}

\begin{Shaded}
\begin{Highlighting}[]
\KeywordTok{cor}\NormalTok{(all_features[}\OperatorTok{!}\StringTok{ }\NormalTok{(}\KeywordTok{names}\NormalTok{(all_features) }\OperatorTok\StringTok{ }\KeywordTok{c}\NormalTok{(}\StringTok{"hash_id"}\NormalTok{,}\StringTok{"module_id"}\NormalTok{,}\StringTok{"deadline"}\NormalTok{))])[,}\StringTok{"unsubmitted"}\NormalTok{]}
\end{Highlighting}
\end{Shaded}

\begin{verbatim}
##                      diff                  miss_ddl                 avg_grade 
##               -0.18346631                0.03308926               -0.14086105 
##            avg_perf_score      not_submitted_by_ddl                num_events 
##               -0.14933670               -0.16619158               -0.11003210 
##    num_video_interactions         num_links_clicked      num_events_past_week 
##               -0.02873552               -0.05731649               -0.10713750 
## num_events_past_three_day       num_events_past_day               unsubmitted 
##               -0.07804353               -0.04961481                1.00000000 
##                    weight 
##                0.01401373
\end{verbatim}

\begin{Shaded}
\begin{Highlighting}[]
\CommentTok{###############################################}
\CommentTok{###############################################}
\end{Highlighting}
\end{Shaded}

\hypertarget{step-5-train-your-models}{%
\section{Step 5: Train your models}\label{step-5-train-your-models}}

\emph{Question 5:} Train a prediction model and iterate on it. You
should try out different algorithms that you have learned so far. You
can go back and check your features and refine them to get better
performance. To check how well you are doing, you should focus on your
training data and compute the F1 score:
\texttt{F1\ =\ 2/{[}(1/recall)+(1/precision){]}}. Report your F1 score
on the training data below (don't forget this!).

\begin{Shaded}
\begin{Highlighting}[]
\CommentTok{############################################### }
\CommentTok{####### }\RegionMarkerTok{BEGIN}\CommentTok{ INPUT: Train and report #########}
\CommentTok{###############################################}

\CommentTok{# Fit  model to training data}
\CommentTok{# add code here}
\CommentTok{### logistic regression}
\NormalTok{m_logreg =}\StringTok{ }\KeywordTok{glm}\NormalTok{(}\KeywordTok{as.factor}\NormalTok{(train}\OperatorTok{$}\NormalTok{unsubmitted) }\OperatorTok{~}\StringTok{ }\NormalTok{. }\OperatorTok{-}\StringTok{ }\NormalTok{hash_id }\OperatorTok{-}\StringTok{ }\NormalTok{module_id }\OperatorTok{-}\StringTok{ }\NormalTok{deadline }\OperatorTok{-}\StringTok{ }\NormalTok{weight, }\DataTypeTok{data =}\NormalTok{ train, }\DataTypeTok{family =} \StringTok{"binomial"}\NormalTok{)}
\CommentTok{# the output are the coefficients:}
\NormalTok{m_logreg}
\end{Highlighting}
\end{Shaded}

\begin{verbatim}
## 
## Call:  glm(formula = as.factor(train$unsubmitted) ~ . - hash_id - module_id - 
##     deadline - weight, family = "binomial", data = train)
## 
## Coefficients:
##               (Intercept)                       diff  
##                 1.007e+00                 -5.423e-06  
##                  miss_ddl                  avg_grade  
##                 2.615e-01                 -1.225e+00  
##            avg_perf_score       not_submitted_by_ddl  
##                -1.456e-01                 -1.982e+01  
##                num_events     num_video_interactions  
##                 4.336e-02                  4.619e-03  
##         num_links_clicked       num_events_past_week  
##                 4.689e-03                 -9.409e-03  
## num_events_past_three_day        num_events_past_day  
##                -1.692e-03                 -9.592e-03  
## 
## Degrees of Freedom: 751 Total (i.e. Null);  740 Residual
## Null Deviance:       289.1 
## Residual Deviance: 180.2     AIC: 204.2
\end{verbatim}

\begin{Shaded}
\begin{Highlighting}[]
\CommentTok{### knn}
\KeywordTok{library}\NormalTok{(class)}
\CommentTok{#predictors = c("avg_perf_score","diff","num_events")}
\CommentTok{#m_knn = knn(train = train[,predictors], test = train[,predictors], cl = train$unsubmitted, k = 3)}
\NormalTok{m_knn =}\StringTok{ }\KeywordTok{knn}\NormalTok{(train[,}\DecValTok{3}\OperatorTok{:}\NormalTok{(}\KeywordTok{ncol}\NormalTok{(train)}\OperatorTok{-}\DecValTok{3}\NormalTok{)], train[,}\DecValTok{3}\OperatorTok{:}\NormalTok{(}\KeywordTok{ncol}\NormalTok{(train)}\OperatorTok{-}\DecValTok{3}\NormalTok{)], train}\OperatorTok{$}\NormalTok{unsubmitted, }\DataTypeTok{k =} \DecValTok{2}\NormalTok{)}
\CommentTok{# the output are the predictions:}
\CommentTok{#m_knn}

\CommentTok{### classification tree}
\KeywordTok{library}\NormalTok{(rpart)}
\CommentTok{# = rpart(unsubmitted ~ avg_grade + more_than_1_possible_point + num_events + num_video_interactions + num_events_past_week + num_events_past_three_day + num_events_past_day, data = train, method = "class")}
\NormalTok{m_class_tree =}\StringTok{ }\KeywordTok{rpart}\NormalTok{(unsubmitted }\OperatorTok{~}\StringTok{ }\NormalTok{diff}\OperatorTok{+}\StringTok{ }\NormalTok{miss_ddl }\OperatorTok{+}\StringTok{ }\NormalTok{avg_grade }\OperatorTok{+}\StringTok{ }\NormalTok{avg_perf_score }\OperatorTok{+}\NormalTok{num_events}\OperatorTok{+}\NormalTok{num_video_interactions}\OperatorTok{+}\StringTok{  }\NormalTok{not_submitted_by_ddl}\OperatorTok{+}\StringTok{ }\NormalTok{num_links_clicked }\OperatorTok{+}\StringTok{ }\NormalTok{num_events_past_week}\OperatorTok{+}\StringTok{ }\NormalTok{num_events_past_three_day }\OperatorTok{+}\StringTok{ }\NormalTok{num_events_past_day , }\DataTypeTok{data =}\NormalTok{ train, }\DataTypeTok{method =} \StringTok{"class"}\NormalTok{)}
\CommentTok{#m_class_tree = rpart(unsubmitted ~ diff + miss_ddl + avg_grade + num_events + num_events_past_week, data = train, method = "class")}
\CommentTok{#m_class_tree = rpart(unsubmitted ~ avg_perf_score + diff + num_events, data = train, method = "class")}
\CommentTok{#m_class_tree = rpart(unsubmitted ~ . - hash_id - module_id - deadline - weight, data = train, method = "class")}
\CommentTok{# the output are the decision trees}
\CommentTok{#m_class_tree}

\CommentTok{# you can even plot it!}
\KeywordTok{plot}\NormalTok{(m_class_tree, }\DataTypeTok{uniform =}\NormalTok{ T)}
\KeywordTok{text}\NormalTok{(m_class_tree, }\DataTypeTok{use.n =}\NormalTok{ F, }\DataTypeTok{all =} \OtherTok{TRUE}\NormalTok{, }\DataTypeTok{cex =} \FloatTok{.8}\NormalTok{)}
\end{Highlighting}
\end{Shaded}

\includegraphics{info4100.project_files/figure-latex/unnamed-chunk-9-1.pdf}

\begin{Shaded}
\begin{Highlighting}[]
\CommentTok{# prune the trees to avoid overfitting by limiting tree complexity}
\CommentTok{#cp_class_tree = m_class_tree$cptable[which.min(m_class_tree$cptable[,"xerror"]),"CP"]}
\CommentTok{#m_class_tree_pruned = prune(m_class_tree, cp = cp_class_tree)}


\CommentTok{### naive bayes}

\KeywordTok{library}\NormalTok{(e1071)}
\CommentTok{#m_nb = naiveBayes(unsubmitted ~ diff + avg_perf_score + num_events, data = train)}
\NormalTok{m_nb =}\StringTok{ }\KeywordTok{naiveBayes}\NormalTok{(}\KeywordTok{as.factor}\NormalTok{(unsubmitted) }\OperatorTok{~}\StringTok{ }\NormalTok{. }\OperatorTok{-}\StringTok{ }\NormalTok{hash_id }\OperatorTok{-}\StringTok{ }\NormalTok{module_id }\OperatorTok{-}\StringTok{ }\NormalTok{deadline }\OperatorTok{-}\StringTok{ }\NormalTok{weight, }\DataTypeTok{data =}\NormalTok{ train, }\DataTypeTok{method =} \StringTok{"class"}\NormalTok{)}
\CommentTok{# the output are a-prior and conditional probabilities}
\CommentTok{#m_nb}

\CommentTok{# Make predictions on the test dataset}
\CommentTok{# logreg: this returns the probability of dropout, so you can set Prob > 0.5 to mean Dropout}
\NormalTok{p_logreg =}\StringTok{ }\KeywordTok{predict}\NormalTok{(m_logreg, }\DataTypeTok{newdata =}\NormalTok{ train, }\DataTypeTok{type =} \StringTok{"response"}\NormalTok{) }\OperatorTok{>}\StringTok{ }\FloatTok{0.5}
\CommentTok{#p_logreg}
\CommentTok{#}
\CommentTok{# knn: this already has the prediction}
\NormalTok{p_knn =}\StringTok{ }\NormalTok{m_knn}
\CommentTok{# class tree}
\NormalTok{p_class_tree =}\StringTok{ }\KeywordTok{predict}\NormalTok{(m_class_tree, }\DataTypeTok{newdata =}\NormalTok{ train, }\DataTypeTok{type =} \StringTok{"class"}\NormalTok{)}
\CommentTok{# naive bayes}
\NormalTok{p_nb =}\StringTok{ }\KeywordTok{predict}\NormalTok{(m_nb, }\DataTypeTok{newdata =}\NormalTok{ train, }\DataTypeTok{type =} \StringTok{"class"}\NormalTok{)}

\NormalTok{cm_logreg =}\StringTok{ }\KeywordTok{table}\NormalTok{(}\DataTypeTok{true =}\NormalTok{ train}\OperatorTok{$}\NormalTok{unsubmitted, }\DataTypeTok{predicted =}\NormalTok{ p_logreg)}
\NormalTok{cm_logreg}
\end{Highlighting}
\end{Shaded}

\begin{verbatim}
##     predicted
## true FALSE TRUE
##    0   711    5
##    1    28    8
\end{verbatim}

\begin{Shaded}
\begin{Highlighting}[]
\NormalTok{cm_knn =}\StringTok{ }\KeywordTok{table}\NormalTok{(}\DataTypeTok{true =}\NormalTok{ train}\OperatorTok{$}\NormalTok{unsubmitted, }\DataTypeTok{predicted =}\NormalTok{ p_knn)}
\NormalTok{cm_class_tree =}\StringTok{ }\KeywordTok{table}\NormalTok{(}\DataTypeTok{true =}\NormalTok{ train}\OperatorTok{$}\NormalTok{unsubmitted, }\DataTypeTok{predicted =}\NormalTok{ p_class_tree)}
\NormalTok{cm_nb =}\StringTok{ }\KeywordTok{table}\NormalTok{(}\DataTypeTok{true =}\NormalTok{ train}\OperatorTok{$}\NormalTok{unsubmitted, }\DataTypeTok{predicted =}\NormalTok{ p_nb)}

\CommentTok{# convenience function for evaluation of confusion matrix}
\NormalTok{cm_eval =}\StringTok{ }\ControlFlowTok{function}\NormalTok{(cm) \{}
    \KeywordTok{list}\NormalTok{(}
        \DataTypeTok{accur =} \KeywordTok{sum}\NormalTok{(}\KeywordTok{diag}\NormalTok{(cm)) }\OperatorTok{/}\StringTok{ }\KeywordTok{sum}\NormalTok{(cm),}
        \DataTypeTok{recall =}\NormalTok{ cm[}\DecValTok{2}\NormalTok{,}\DecValTok{2}\NormalTok{] }\OperatorTok{/}\StringTok{ }\KeywordTok{sum}\NormalTok{(cm[}\DecValTok{2}\NormalTok{,]),}
        \DataTypeTok{precision =}\NormalTok{ cm[}\DecValTok{2}\NormalTok{,}\DecValTok{2}\NormalTok{] }\OperatorTok{/}\StringTok{ }\KeywordTok{sum}\NormalTok{(cm[,}\DecValTok{2}\NormalTok{]),}
        \DataTypeTok{F1 =} \DecValTok{2} \OperatorTok{/}\StringTok{ }\NormalTok{(}\DecValTok{1}\OperatorTok{/}\NormalTok{(cm[}\DecValTok{2}\NormalTok{,}\DecValTok{2}\NormalTok{] }\OperatorTok{/}\StringTok{ }\KeywordTok{sum}\NormalTok{(cm[}\DecValTok{2}\NormalTok{,])) }\OperatorTok{+}\StringTok{ }\DecValTok{1}\OperatorTok{/}\NormalTok{(cm[}\DecValTok{2}\NormalTok{,}\DecValTok{2}\NormalTok{] }\OperatorTok{/}\StringTok{ }\KeywordTok{sum}\NormalTok{(cm[,}\DecValTok{2}\NormalTok{])))}
\NormalTok{    )}
\NormalTok{\}}

\KeywordTok{cm_eval}\NormalTok{(cm_logreg)}
\end{Highlighting}
\end{Shaded}

\begin{verbatim}
## $accur
## [1] 0.956117
## 
## $recall
## [1] 0.2222222
## 
## $precision
## [1] 0.6153846
## 
## $F1
## [1] 0.3265306
\end{verbatim}

\begin{Shaded}
\begin{Highlighting}[]
\KeywordTok{cm_eval}\NormalTok{(cm_knn)}
\end{Highlighting}
\end{Shaded}

\begin{verbatim}
## $accur
## [1] 0.962766
## 
## $recall
## [1] 0.6944444
## 
## $precision
## [1] 0.5952381
## 
## $F1
## [1] 0.6410256
\end{verbatim}

\begin{Shaded}
\begin{Highlighting}[]
\KeywordTok{cm_eval}\NormalTok{(cm_class_tree)}
\end{Highlighting}
\end{Shaded}

\begin{verbatim}
## $accur
## [1] 0.9694149
## 
## $recall
## [1] 0.5277778
## 
## $precision
## [1] 0.76
## 
## $F1
## [1] 0.6229508
\end{verbatim}

\begin{Shaded}
\begin{Highlighting}[]
\KeywordTok{cm_eval}\NormalTok{(cm_nb)}
\end{Highlighting}
\end{Shaded}

\begin{verbatim}
## $accur
## [1] 0.650266
## 
## $recall
## [1] 0.9444444
## 
## $precision
## [1] 0.1152542
## 
## $F1
## [1] 0.2054381
\end{verbatim}

\begin{Shaded}
\begin{Highlighting}[]
\KeywordTok{library}\NormalTok{(caret)}
\end{Highlighting}
\end{Shaded}

\begin{verbatim}
## Loading required package: lattice
\end{verbatim}

\begin{verbatim}
## 
## Attaching package: 'caret'
\end{verbatim}

\begin{verbatim}
## The following object is masked from 'package:purrr':
## 
##     lift
\end{verbatim}

\begin{Shaded}
\begin{Highlighting}[]
\CommentTok{#all_features$weight}
\NormalTok{fitControl <-}\StringTok{ }\KeywordTok{trainControl}\NormalTok{(}\DataTypeTok{method =} \StringTok{"none"}\NormalTok{)}

\CommentTok{#set.seed(825)}
\NormalTok{knnFit4 <-}\StringTok{ }\KeywordTok{train}\NormalTok{(}\KeywordTok{as.factor}\NormalTok{(unsubmitted) }\OperatorTok{~}\StringTok{ }\NormalTok{. , }\DataTypeTok{data =}\NormalTok{ train[,}\OperatorTok{!}\KeywordTok{colnames}\NormalTok{(train) }\OperatorTok\StringTok{ }\KeywordTok{c}\NormalTok{(}\StringTok{"hash_id"}\NormalTok{,}\StringTok{"module_id"}\NormalTok{,}\StringTok{"deadline"}\NormalTok{,}\StringTok{"weight"}\NormalTok{)], }
                 \DataTypeTok{method =} \StringTok{"knn"}\NormalTok{, }
                 \DataTypeTok{trControl =}\NormalTok{ fitControl, }
                 \DataTypeTok{weights =}\NormalTok{ train}\OperatorTok{$}\NormalTok{weight}
\NormalTok{                 )}
\NormalTok{knnFit4}
\end{Highlighting}
\end{Shaded}

\begin{verbatim}
## k-Nearest Neighbors 
## 
## 752 samples
##  11 predictor
##   2 classes: '0', '1' 
## 
## No pre-processing
## Resampling: None
\end{verbatim}

\begin{Shaded}
\begin{Highlighting}[]
\CommentTok{#predict(gbmFit4, newdata = as.factor(head(test)))}
\NormalTok{p_knn =}\StringTok{ }\KeywordTok{predict}\NormalTok{(knnFit4, }\DataTypeTok{newdata =}\NormalTok{ train)}
\NormalTok{cm_knn =}\StringTok{ }\KeywordTok{table}\NormalTok{(}\DataTypeTok{true =}\NormalTok{ train}\OperatorTok{$}\NormalTok{unsubmitted, }\DataTypeTok{predicted =}\NormalTok{ p_knn)}
\KeywordTok{cm_eval}\NormalTok{(cm_knn)}
\end{Highlighting}
\end{Shaded}

\begin{verbatim}
## $accur
## [1] 0.9547872
## 
## $recall
## [1] 0.08333333
## 
## $precision
## [1] 0.75
## 
## $F1
## [1] 0.15
\end{verbatim}

\begin{Shaded}
\begin{Highlighting}[]
\NormalTok{p_knn =}\StringTok{ }\KeywordTok{predict}\NormalTok{(knnFit4, }\DataTypeTok{newdata =}\NormalTok{ test)}
\NormalTok{cm_knn =}\StringTok{ }\KeywordTok{table}\NormalTok{(}\DataTypeTok{true =}\NormalTok{ test}\OperatorTok{$}\NormalTok{unsubmitted, }\DataTypeTok{predicted =}\NormalTok{ p_knn)}
\KeywordTok{cm_eval}\NormalTok{(cm_knn)}
\end{Highlighting}
\end{Shaded}

\begin{verbatim}
## $accur
## [1] 0.9088472
## 
## $recall
## [1] 0.0952381
## 
## $precision
## [1] 0.1176471
## 
## $F1
## [1] 0.1052632
\end{verbatim}

\begin{Shaded}
\begin{Highlighting}[]
\CommentTok{###############################################}
\CommentTok{###############################################}
\end{Highlighting}
\end{Shaded}

\hypertarget{step-6-test-your-model}{%
\section{Step 6: Test your model}\label{step-6-test-your-model}}

\emph{Question 6:} Using the model that you arrived at, predict on the
held-out test data and report your final F1 score. Typically, you would
only do this once at the very end, but for this project it is actually
rather hard to do well on the test set, so you can try your model
(sparingly to avoid overfitting too much) on the test data to compute
the testing F1 score.

\begin{Shaded}
\begin{Highlighting}[]
\CommentTok{############################################### }
\CommentTok{####### }\RegionMarkerTok{BEGIN}\CommentTok{ INPUT: Test and report ##########}
\CommentTok{###############################################}

\CommentTok{# Make predictions on the test dataset}
\CommentTok{# logreg: this returns the probability of dropout, so you can set Prob > 0.5 to mean Dropout}
\CommentTok{#m_logreg = glm(unsubmitted ~ . - hash_id - module_id - deadline, data = train, family = "binomial")}
\CommentTok{#p_logreg = predict(m_logreg, newdata = test, type = "response") > 0.5}
\NormalTok{m_logreg =}\StringTok{ }\KeywordTok{glm}\NormalTok{(unsubmitted }\OperatorTok{~}\StringTok{ }\NormalTok{. , }\DataTypeTok{data =}\NormalTok{ train[,}\OperatorTok{!}\KeywordTok{colnames}\NormalTok{(train) }\OperatorTok\StringTok{ }\KeywordTok{c}\NormalTok{(}\StringTok{"hash_id"}\NormalTok{,}\StringTok{"module_id"}\NormalTok{,}\StringTok{"deadline"}\NormalTok{,}\StringTok{"weight"}\NormalTok{)], }\DataTypeTok{family =} \StringTok{"binomial"}\NormalTok{)}
\NormalTok{m_logreg}\OperatorTok{$}\NormalTok{xlevels[[}\StringTok{"module_id"}\NormalTok{]] =}\StringTok{ }\KeywordTok{union}\NormalTok{(m_logreg}\OperatorTok{$}\NormalTok{xlevels[[}\StringTok{"module_id"}\NormalTok{]], }\KeywordTok{levels}\NormalTok{(test}\OperatorTok{$}\NormalTok{module_id))}
\NormalTok{p_logreg =}\StringTok{ }\KeywordTok{predict}\NormalTok{(m_logreg, }\DataTypeTok{newdata =}\NormalTok{ test, }\DataTypeTok{type =} \StringTok{"response"}\NormalTok{) }\OperatorTok{>}\StringTok{ }\FloatTok{0.5}

\CommentTok{#}
\CommentTok{# knn: this already has the prediction}
\NormalTok{p_knn =}\StringTok{ }\KeywordTok{knn}\NormalTok{(train[,}\DecValTok{3}\OperatorTok{:}\NormalTok{(}\KeywordTok{ncol}\NormalTok{(train)}\OperatorTok{-}\DecValTok{3}\NormalTok{)], test[,}\DecValTok{3}\OperatorTok{:}\NormalTok{(}\KeywordTok{ncol}\NormalTok{(train)}\OperatorTok{-}\DecValTok{3}\NormalTok{)], train}\OperatorTok{$}\NormalTok{unsubmitted, }\DataTypeTok{k =} \DecValTok{5}\NormalTok{)}
\CommentTok{# class tree}
\NormalTok{p_class_tree =}\StringTok{ }\KeywordTok{predict}\NormalTok{(m_class_tree, }\DataTypeTok{newdata =}\NormalTok{ test, }\DataTypeTok{type =} \StringTok{"class"}\NormalTok{)}
\CommentTok{# naive bayes}
\NormalTok{p_nb =}\StringTok{ }\KeywordTok{predict}\NormalTok{(m_nb, }\DataTypeTok{newdata =}\NormalTok{ test, }\DataTypeTok{type =} \StringTok{"class"}\NormalTok{)}
\CommentTok{# add code here}

\CommentTok{# here is the confusion matrix for the logreg model:}
\NormalTok{cm_logreg =}\StringTok{ }\KeywordTok{table}\NormalTok{(}\DataTypeTok{true =}\NormalTok{ test}\OperatorTok{$}\NormalTok{unsubmitted, }\DataTypeTok{predicted =}\NormalTok{ p_logreg)}
\NormalTok{cm_knn =}\StringTok{ }\KeywordTok{table}\NormalTok{(}\DataTypeTok{true =}\NormalTok{ test}\OperatorTok{$}\NormalTok{unsubmitted, }\DataTypeTok{predicted =}\NormalTok{ p_knn)}
\NormalTok{cm_class_tree =}\StringTok{ }\KeywordTok{table}\NormalTok{(}\DataTypeTok{true =}\NormalTok{ test}\OperatorTok{$}\NormalTok{unsubmitted, }\DataTypeTok{predicted =}\NormalTok{ p_class_tree)}

\NormalTok{cm_nb =}\StringTok{ }\KeywordTok{table}\NormalTok{(}\DataTypeTok{true =}\NormalTok{ test}\OperatorTok{$}\NormalTok{unsubmitted, }\DataTypeTok{predicted =}\NormalTok{ p_nb)}

\NormalTok{cm_logreg}
\end{Highlighting}
\end{Shaded}

\begin{verbatim}
##     predicted
## true FALSE TRUE
##    0   343    9
##    1    14    7
\end{verbatim}

\begin{Shaded}
\begin{Highlighting}[]
\KeywordTok{cm_eval}\NormalTok{(cm_logreg)}
\end{Highlighting}
\end{Shaded}

\begin{verbatim}
## $accur
## [1] 0.9383378
## 
## $recall
## [1] 0.3333333
## 
## $precision
## [1] 0.4375
## 
## $F1
## [1] 0.3783784
\end{verbatim}

\begin{Shaded}
\begin{Highlighting}[]
\NormalTok{cm_knn}
\end{Highlighting}
\end{Shaded}

\begin{verbatim}
##     predicted
## true   0   1
##    0 336  16
##    1  19   2
\end{verbatim}

\begin{Shaded}
\begin{Highlighting}[]
\KeywordTok{cm_eval}\NormalTok{(cm_knn)}
\end{Highlighting}
\end{Shaded}

\begin{verbatim}
## $accur
## [1] 0.9061662
## 
## $recall
## [1] 0.0952381
## 
## $precision
## [1] 0.1111111
## 
## $F1
## [1] 0.1025641
\end{verbatim}

\begin{Shaded}
\begin{Highlighting}[]
\NormalTok{cm_class_tree}
\end{Highlighting}
\end{Shaded}

\begin{verbatim}
##     predicted
## true   0   1
##    0 343   9
##    1  16   5
\end{verbatim}

\begin{Shaded}
\begin{Highlighting}[]
\KeywordTok{cm_eval}\NormalTok{(cm_class_tree)}
\end{Highlighting}
\end{Shaded}

\begin{verbatim}
## $accur
## [1] 0.9329759
## 
## $recall
## [1] 0.2380952
## 
## $precision
## [1] 0.3571429
## 
## $F1
## [1] 0.2857143
\end{verbatim}

\begin{Shaded}
\begin{Highlighting}[]
\NormalTok{cm_nb }
\end{Highlighting}
\end{Shaded}

\begin{verbatim}
##     predicted
## true   0   1
##    0 271  81
##    1   4  17
\end{verbatim}

\begin{Shaded}
\begin{Highlighting}[]
\KeywordTok{cm_eval}\NormalTok{(cm_nb)}
\end{Highlighting}
\end{Shaded}

\begin{verbatim}
## $accur
## [1] 0.772118
## 
## $recall
## [1] 0.8095238
## 
## $precision
## [1] 0.1734694
## 
## $F1
## [1] 0.2857143
\end{verbatim}

\begin{Shaded}
\begin{Highlighting}[]
\CommentTok{###############################################}
\CommentTok{###############################################}
\end{Highlighting}
\end{Shaded}

\hypertarget{step-7-report}{%
\section{Step 7: Report}\label{step-7-report}}

\emph{Question 7:} As a team, write a brief report. Imagine your
supervisor asked you to investigate the possibility of an early warning
system. She would like to know what model to use, what features are
important, and most importantly how well it would work. Given what
you've learned, would you recommend implementing the system? Write your
report answering the above questions here:

\%\#\#\#\#\#\#\#\# BEGIN INPUT: Summarize findings
\#\#\#\#\#\#\#\#\#\#\#\#

In question 2, we created the outcome indicator based on
is.na(first\_attempted\_timestamp), that is, if the students actually
submitted the homework before the actual deadline or not since the edX
system does not accept submissions after the deadline. There are 13
assignments that are used for this analysis. There are two types of
assignments: 9 of which are reading comprehension, reading reflection,
and video reflection, which all have a possible grade of 1, and 4 of
which are the individual and group homework with possible grades varying
from 8 to 19 points.

After testing many different factors, we included the ones with the
highest correlation values in our model. Despite these features being
the most important to our model compared to the many others that we
engineered, they still had low correlation values, the highest only
about 0.2. We learned that predicting which students will not submit an
assignment through this system is difficult because overall, the
possible factors have low correlation values, resulting in F1 values
below 0.4 for the models. This is true even for the logistic regression
model, our best performing model with an F1 score of 0.3783784.

Given our results, we would in fact not recommend implementing the
system. Our most important features included the average difference in
time between when a student submitted and the calculated deadline, the
number of clickstream events they generated, their average grade, how
often on average they got a perfect score on an assignment, and whether
or not they submitted assignments on time in the past.

One possible reason that our system did not work well might be the NA
values in the original datasets and in the features we engineered. The
NA values in first\_attempted\_timestamp were used to determine the
outcome variable. In some of our features, we calculated the difference
between the deadline and first\_attempted\_timestamp, which sometimes
resulted in NA values for those features. Some models, such as the Naive
Bayes, are able to still make predictions even when missing attributes
are present. Thus, the impact of missing data largely relates to the
choice of learning algorithm. We attempted to work around this issue by
manually changing the null values to 0 where we thought it made sense to
do so, but this altered the fit of the models to the data. We also
considered replacing the missing values with a ``normalized'' value
(i.e.~mean of known values), but this would not have worked for binary
features where we were seeing the NA values.

In evaluating the obstacles we faced when creating this model, we feel
we had a particularly difficult time relating the data back to the
corresponding due dates. Particularly because we had no basis on which
events were related to which assignment, we were making these
classifications purely based on due date (which was an assumed value)
and the timestamp of the event. While these methods were useful, future
models may be more successful with more clear data relating to the due
date and the relevant events to each assignment.

While creating the models, we made some other observations we believe
may be helpful if someone were to look into designing a similar system
in the future. When we generated the percentage of unsubmitted
assignments for each module we noticed one assignment, the only video
reflection assignment, has the highest percentage of being unsubmitted
relative to other assignments. Upon examining it more closely, we
concluded that this might be because this assignment was due on a Friday
followed by the first group project, an unusual time that students might
forget. It is possible that our model is less effective because unique
circumstances such as this may be influencing why a student doesn't
submit more often, as opposed to other more predictable situations such
as a student consistently struggling in the course.

\%\#\#\#\#\#\#\#\#\#\#\#\#\#\#\#\#\#\#\#\#\#\#\#\#\#\#\#\#\#\#\#\#\#\#\#\#\#\#\#\#\#\#\#\#\#\#\#

\hypertarget{submit-project}{%
\section{Submit Project}\label{submit-project}}

This is the end of the project. Please \textbf{Knit a Word doc report}
that shows both the R code and R output and upload it on the EdX
platform. EACH TEAM MEMBER NEEDS TO SUBMIT THE REPORT ON EDX TO GET
CREDIT.

\end{document}
